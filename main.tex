% !TEX program = xelatex
\documentclass[12pt,oneside]{report}

% -------- Packages --------
% للطباعة مع تجليد يسار مطابق للصورة (A4 + هامش تجليد 8mm)
\usepackage[a4paper,left=3cm,right=2.5cm,top=3cm,bottom=3cm,bindingoffset=8mm]{geometry}

\usepackage{graphicx}
\usepackage{tabularx}
\usepackage{longtable}
\usepackage{booktabs}
\usepackage{array}
\usepackage{caption}
\usepackage{float}
\usepackage{setspace}
\usepackage{titlesec}
\usepackage{tocloft}
\usepackage{xcolor}
\usepackage{enumitem}
\usepackage{fontspec}
\usepackage[hidelinks]{hyperref}
\usepackage{polyglossia}
\usepackage{ragged2e} % للضبط الكامل داخل الأعمدة

\setmainlanguage{english}
\setotherlanguage{arabic}

% Fonts
\setmainfont{Times New Roman}
\newfontfamily\arabicfont[Script=Arabic]{Times New Roman}
\newfontfamily\arialfont{Arial}

% ----- FIX: dummy definition for \tbl_save_outer_table_cols: to avoid bidi+tabularx error -----
\ExplSyntaxOn
\cs_if_exist:NF \tbl_save_outer_table_cols:
  {
    \cs_new_protected:Npn \tbl_save_outer_table_cols: {}
  }
\ExplSyntaxOff
% ----------------------------------------------------------------------- %

% Line spacing (default 1.5 as requested)
\onehalfspacing

% Section formatting (clean look)
\titleformat{\chapter}{\normalfont\Large\bfseries}{\thechapter}{1em}{}
\titleformat{\section}{\normalfont\large\bfseries}{\thesection}{0.8em}{}
\titleformat{\subsection}{\normalfont\normalsize\bfseries}{\thesubsection}{0.6em}{}

% TOC spacing a bit tighter
\setlength{\cftbeforechapskip}{6pt}
\setlength{\cftbeforesecskip}{2pt}

% Figure/Table captions (حجم أصغر وخط عريض لعنوان الكابتشن)
\captionsetup{font=small,labelfont=bf}

% ====== الإعداد المهم للجداول ======
% 1) إلغاء "Table X:" من الكابتشن داخل النص، وطباعة محتوى \caption كما هو
\captionsetup[table]{labelformat=empty,labelsep=none}
% الأشكال بدون "Figure X:"
\captionsetup[figure]{labelformat=empty,labelsep=none}
% 2) في قائمة الجداول (List of Tables): إخفاء رقم الجدول التلقائي
%    وترك نص الكابتشن كما هو (الذي يحتوي "Table 2.7.1 Login") مع نقاط وقائدات للصفحات
\renewcommand{\cfttabpresnum}{}   % لا تطبع الرقم قبل العنوان
\renewcommand{\cfttabaftersnum}{} % ولا أي فاصل بعده
\setlength{\cfttabnumwidth}{0pt}  % عدم حجز عرض لرقم الجدول
\renewcommand{\cfttableader}{\cftdotfill{\cftdotsep}} % تأكيد نقاط القيادة

% ---- Placeholders ----
\newcommand{\placeholderfigure}[1]{%
  \begin{figure}[H]
    \centering
    \fbox{\parbox[c][6cm][c]{0.9\textwidth}{\centering \textbf{PLACEHOLDER:} #1\\[6pt] \textit{Insert figure file here}}}
    \caption{#1}
  \end{figure}
}
\newcommand{\pastehere}[1]{\noindent\textcolor{gray}{\textit{[PASTE HERE: #1]}}\par}

% ---- Handy centered-column environments (جاهزة للصق) ----
\newenvironment{ENcentercol}[1][0.8\textwidth]{%
  \begin{center}\begin{minipage}{#1}\justifying
}{%
  \end{minipage}\end{center}
}
\newenvironment{ARcentercol}[1][0.8\textwidth]{%
  \begin{Arabic}\begin{center}\begin{minipage}{#1}\justifying
}{%
  \end{minipage}\end{center}\end{Arabic}
}
% --- Helper: render multiple paragraphs as one visual block (no extra spacing/indent) ---
\newenvironment{singlepara}{%
  \begingroup
  \setlength{\parindent}{0pt}%
  \setlength{\parskip}{0pt}%
}{\par\endgroup}

% --- Chapter title on its own page (numbered + in TOC) ---
\newcommand{\chaptertitlepage}[1]{%
  \clearpage
  \refstepcounter{chapter}% increase chapter counter
  \setcounter{section}{0}% reset section numbering like \chapter would
  \setcounter{subsection}{0}% reset subsection numbering
  \addcontentsline{toc}{chapter}{Chapter \thechapter: #1}% TOC entry
  \markboth{Chapter \thechapter: #1}{}
  \thispagestyle{empty}
  \vspace*{\fill}
  \begin{center}
    {\Huge\bfseries Chapter \thechapter: #1\par}
  \end{center}
  \vspace*{\fill}
  \clearpage
}


\begin{document}

% -------------------- Title Page --------------------
\begin{titlepage}
\thispagestyle{empty}
\centering

% --- University Logo placeholder (استبدله بـ \includegraphics) ---
\vspace*{6mm}
\includegraphics[width=0.37\textwidth]{ppu_logo.png}
\\[10mm]

{\Large \textbf{Palestine Polytechnic University}}\\[6pt]
{\large College of Information Technology and Computer Engineering}\\[18pt]

{\LARGE \textbf{Project Title:}}\\[6pt]
{\Large \textbf{Machine-Learning-Based Land-Price Prediction System}}\\[20pt]

{\large Team Members:}\\
{\large Mohammad Alqadi \quad|\quad Mohammad Alamlah}\\[8pt]

{\large Supervisor: Dr.~Hashem Altamimi}\\[28pt]

\vfill

\end{titlepage}

% -------------------- Front Matter --------------------
\pagenumbering{roman}

% ===== Dedication + Acknowledgement on SAME page (جاهزة للصق) =====
\begin{center}\Large\bfseries \textarabic{إهداء}\end{center}
\addcontentsline{toc}{chapter}{Dedication / \textarabic{إهداء}}
\begin{ARcentercol}
\textbf{إلى والدينا}\\
بفصاحة القلب وبكل احترام وتقدير، نتوجه إليكم برسالة ممتلئة بعمق المشاعر وارتفاع الجلال. إن ما نحمله في قلوبنا من امتنان ومودة لا يمكن وصفه بكلمات بسيطة، فأنتما الركيزة الثابتة التي بنينا عليها حياتنا، والشمعة الساطعة التي أضاءت دربنا في ظلمة الليالي..

\medskip
{\centering \large … \par}
\medskip

\textbf{إلى أصدقائنا}\\
بكل احترام وتقدير، نرفع لكم تحية الود والاعتزاز، فأنتم أصدقاؤنا الأوفياء والرفاق المخلصون. لقد كنتم دائماً العون والسند في السراء والضراء، والصخرة الصلبة التي نستند إليها في عبور مياه الحياة العميقة. فشكرًا لكم على كل لحظة قضيناها معًا، وعلى كل دعمكم اللامحدود وتضحياتكم الجليلة.

\medskip

\end{ARcentercol}

\vspace{2em}
\clearpage
\phantomsection

\begin{center}\Large\bfseries \textarabic{شكر وتقدير}\end{center}
\addcontentsline{toc}{chapter}{Acknowledgement / \textarabic{شكر وتقدير}}
\begin{ARcentercol}
إلى أساتذتنا الكرام، نتقدم بأخلص الشكر والتقدير على الجهود الجبارة التي بذلتموها خلال سنواتنا في الدراسة. لقد كنتم قدوةً ومصدرَ إلهامٍ لنا، وساهمتم بشكل كبير في تشكيل مستقبلنا الأكاديمي والمهني.

\medskip

نود أن نخصّ بالشكر \textbf{الدكتور هاشم هشام التميمي} على تفانيه وإرشاده القيم، وعلى كل العلم والمعرفة التي شاركنا بها. لقد كنتم داعمًا لنا في كل خطوة نخطوها في طريقنا التعليمي.

\medskip

نشكركم على صبركم الذي لا يُضاهى واحتوائكم لنا في كل الظروف، وعلى توفير بيئة تعليمية محفِّزة ومليئة بالتشجيع. إن مساهماتكم \textbf{لن تُنسى}، وستظل خالدة في ذاكرتنا.

\medskip

ونخصّ بالشكر \textbf{المخمّن العقاري قيس ادعيس} على تزويدنا ببيانات ميدانية واقعية وإرشادات مهنية أسهمت مباشرةً في بناء قاعدة البيانات واختبار النموذج.

\medskip

ونتوجه بالشكر أيضًا إلى \textbf{جامعة بوليتكنك فلسطين} على توفير المرافق التعليمية المتميزة والخدمات التي ساعدتنا على تحقيق أهدافنا الأكاديمية بنجاح.

\medskip

ندعو الله أن يجزيكم خير الجزاء وأن يوفقكم في كل ما تسعون إليه من خير وتطور في خدمة العلم والتعليم.
\end{ARcentercol}


% ===== End (same page) =====

% ===== English Abstract (centered title, own page) =====
\clearpage
\phantomsection
\addcontentsline{toc}{chapter}{Abstract}
\begin{center}
  \Large\bfseries Abstract
\end{center}

\begin{ENcentercol}
{\arialfont\singlespacing\justifying
\begin{singlepara}
In the era of artificial intelligence and technological  advancements, the process of predicting ( or estimating ) land prices is still implemented using traditional methods that rely on human estimation, which makes it  prone to bias and inconsistency in results.
In response to these challenges, this project aims to develop an intelligent system that depends on machine learning and real-world data  to estimate land prices more objectively and more accurately, and in less time compared to traditional methods. The town of Bani Na'im, located in the Hebron Governorate in Palestine, was chosen as an experimental area to apply the system because there is enough available data about its lands, and local land appraisers cooperated by providing us with this data.

The system is designed with an interactive user interface that provides a form with fields to enter  land features such as area, location, political classification, and other influencing factors, to provide an immediate estimated price for the user.
The regression tree algorithm was chosen for the project in its early stage due to its simplicity and efficiency in dealing with a limited amount of data, which is the case with the data currently available.
The used data included both numerical and categorical features, the model was trained on this data to estimate  the price based on the entered factors.
The data was collected from various sources, the most important being the land appraisers from Bani Na'im as well as referring to official maps and structural plans to extract important information about the lands, such as their location, classification, shape, and price, so they can be manually entered into the system. The diverse sources helped build a realistic database,  and reinforced the authenticity of the model and its relevance to the practical field. Although the available data is limited, the model was optimized to achieve a balance between precision and speed, which makes it an effective helping tool for the decisions of real estate appraisers, since they are the main users who benefit from it.
This project represents the first step in automating the process of real estate valuation, and it is planned to develop it in the future using more advanced algorithms such as CatBoost and Random Forest, to keep up with the  increasing volume and variety of available data.
\end{singlepara}
}
\end{ENcentercol}

% ===== Arabic Abstract (centered title, own page) =====
\clearpage
\phantomsection
\addcontentsline{toc}{chapter}{\textarabic{الخلاصة}}
\begin{center}
  \Large\bfseries \textarabic{الخلاصة}
\end{center}

\begin{ARcentercol}
\singlespacing
في عصر نهضة الذكاء الاصطناعي والتطور الملحوظ لا تزال عملية تخمين (أو تثمين) أسعار الأراضي تُنفّذ بأساليب تقليدية تعتمد على التقدير البشري؛ فهذا يجعلها عرضة للتحيّز والتفاوت في النتائج.
استجابةً لهذه التحديات، يهدف هذا المشروع إلى تطوير نظام ذكي يعتمد على تعلم الآلة لتقدير أسعار الأراضي بموضوعية ودقة أعلى، وفي زمن أقل مقارنةً بالأساليب التقليدية، وذلك بالاعتماد على بيانات واقعية تم جمعها. وقد تم اختيار بلدة بني نعيم الواقعة في محافظة الخليل، فلسطين، كنموذج أولي لتطبيق النظام. نظراً لتوفر بيانات كافية حول أراضيها، وتعاون مخمني الأراضي من خلال تزويدنا بها.
تم تصميم النظام بواجهة مستخدم تفاعلية تتيح إدخال خصائص الأرض مثل المساحة، والموقع، والتصنيف السياسي، وغيرها من العوامل المؤثرة، ليحصل المستخدم على سعر تقديري فوري.
اعتمد المشروع في مرحلته الأولى خوارزمية شجرة الانحدار، نظرًا لبساطتها وقدرتها على التعامل بكفاءة مع أحجام بيانات محدودة، كما هو الحال مع البيانات المتوفرة حاليًا.
تضمنت البيانات المستخدمة خصائص عددية وأخرى تصنيفية وتم تدريب النموذج عليها لتقدير السعر بناءً على العوامل المدخلة.
وقد تم تحصيل هذه البيانات من مصادر متنوعة أهمها مخمنو الأراضي في بلدة بني نعيم، بجانب الرجوع إلى خرائط رسمية ومخططات هيكلية لاستخلاص معلومات تنظيمية عن الأراضي مثل موقعها وتصنيفها وشكلها، وسعرها، وذلك لإدخالها يدويًا إلى النظام. ساعد هذا التنوع في بناء قاعدة بيانات واقعية، وعزّز من موثوقية النموذج وارتباطه بالميدان العملي. ورغم محدودية البيانات المتوفرة، تم ضبط النموذج لتحقيق توازن فعّال بين الدقة وسرعة التنفيذ، مما يجعله أداة مساعدة فعالة لقرارات المخمّنين العقاريين، كونهم الفئة المستفيدة منه بشكل رئيسي.
يمثّل هذا المشروع الخطوة الأولى في أتمتة عملية التثمين العقاري، ويُخطط لتطويره لاحقًا باستخدام خوارزميات أكثر تقدمًا مثل CatBoost وRandom Forest، بما يتماشى مع الازدياد في حجم وتنوع البيانات المتوفرة.
\end{ARcentercol}



% ===== TOC / LOT / LOF each on its own page =====
\clearpage
\tableofcontents

\clearpage
{\renewcommand{\numberline}[1]{}\listoftables}

\clearpage
{\renewcommand{\numberline}[1]{}\listoffigures}


\clearpage
\pagenumbering{arabic}

% -------------------- Chapter 1 --------------------
\chaptertitlepage{Introduction}

\section{Overview}
This chapter introduces the main elements of the project. It begins with the idea of the project, then goes to its importance, followed by the goals of the project, the scope and limitations, the theoretical background, and finally the chosen algorithm and alternatives.

\section{Idea of the Project}
The main idea of the project is building a web application for an intelligent system that is capable of accurately predicting the land prices in the town of Bani Na'im.
In order for the system to predict accurately, it will depend on the techniques of artificial intelligence (AI) and machine learning (ML).
The machine learning model will be trained by providing for it all the major factors affecting land prices, these factors include: area, distance to main roads and markets, availability of water and electricity supplies, among others. By feeding the machine learning algorithm with this data the project aims to provide a faster, more accurate, and more transparent alternative to traditional land valuation methods.
The project intends to benefit the appraisers and help them make objective and data driven decisions.

\section{Importance}
Due to the frequent transactions in the area, the need for a faster and more efficient pricing method is growing, and one of the main advantages of this project's AI-powered approach is speed, while traditional/manual methods can take several hours to evaluate the price, the trained machine learning model can do it in seconds.
Another need is reducing subjectivity, it is crucial to avoid the human bias in the field of land price evaluation because human bias in land valuation can shift prices by thousands of shekels. The project eliminates such bias by relying on data and algorithms alone, ensuring objective, data-driven, and transparent predictions.

\section{Goals of the Project}
The main goal of the project is to develop a machine learning model capable of accurately estimating land prices in Bani Na’im and to achieve this goal, the project has the following objectives:

\begin{enumerate}[leftmargin=*, label=\arabic*-, align=left]
    \item Data collection: Gather all relevant land data like the area, location, suitability for agriculture and more.
    \item Data cleaning: After collecting the raw data, data cleaning is performed, where the data’s quality will be enhanced by removing duplicate and irrelevant data entries, and correcting inconsistencies.
    \item Model development: Train machine learning algorithms with the cleaned data and compare them to select the most suitable algorithm in predicting the prices.
    \item Model evaluation: Test the model and evaluate it by comparing the results of the model with the actual results of traditional pricing methods.
    \item Tool implementation: Design a user-friendly website as a tool for the land appraisers.
\end{enumerate}

The project aims to increase the efficiency and transparency of the land price estimations as well as making the process of price estimation easier for the appraisers.

\section{Scope and Limitations}
This project aims to predict land prices in the town of Bani Na'im using machine learning techniques depending on available real Bani Na'im land data.
The scope of the project includes developing a predictive machine learning model that predicts land prices depending on features like:

\begin{itemize}[leftmargin=*]
    \item Location
    \item Political classification of the land (area A, B, or C)
    \item Intended land use (e.g., residential, commercial)
    \item Land area
    \item Availability of infrastructure
    \item Proximity to essential public services (e.g., hospitals)
\end{itemize}

The project also involves designing a user-friendly interface that allows authorized users such as land appraisers, system admins, and data scientists to use the system — everyone as allowed to.

On the other hand, the project faces many challenges that may affect the accuracy of the prediction. The accuracy depends directly on data quality and completeness in addition to the used model. The most important limitations are:

\begin{itemize}[leftmargin=*]
    \item \textbf{Limited data availability:} The collected data may be outdated or incomplete, and some land prices may not be documented.
    \item \textbf{Assumption of data representativeness:} This project assumes that the available data reflects typical land characteristics in Bani Na’im.
    \item \textbf{Geographic limitation:} The model is specifically designed for lands just in Bani Na’im.
    \item \textbf{Not considering all external factors:} The model does not account for sudden market shifts, and in Palestine, Palestinians are vulnerable to forced displacement at any moment, which could cause a sudden gap in land prices.
    \item \textbf{Limited time:} Because of the limited time that we have, our team was not able to try many machine learning algorithms to choose the best one that validated our project.
\end{itemize}

\section{Background}
\subsection{Artificial Intelligence (AI)}
Artificial Intelligence (AI) is one of the most significant fields in modern computer science. It aims to develop intelligent systems capable of performing tasks that traditionally require human intelligence, such as reasoning, decision-making, pattern recognition, and prediction. AI systems rely on processing large datasets, extracting meaningful relationships, and generating insights that enable faster, more accurate, and more objective decisions compared to traditional manual approaches. In recent years, AI applications have expanded across numerous domains, including the real estate sector, where AI contributes to producing reliable, data-driven land and property price estimations.

\subsection{Machine Learning (ML)}
Machine Learning (ML) is a core discipline within AI that focuses on constructing models capable of learning automatically from data rather than being explicitly programmed for every possible case. ML models analyze historical data, discover patterns and relationships between input features and target variables, and utilize this knowledge to make predictions on new, unseen data.

Machine learning approaches are commonly categorized into three main types:
\begin{itemize}[leftmargin=*]
    \item \textbf{Supervised Learning:} The model is trained using labeled data that includes both input features and their corresponding correct outputs. This enables the model to learn the mapping between inputs and outputs.
    \item \textbf{Unsupervised Learning:} The model learns from unlabeled data, aiming to uncover hidden structures, clusters, or patterns within the dataset.
    \item \textbf{Reinforcement Learning:} The model learns by interacting with an environment, receiving feedback in the form of rewards or penalties, and improving its performance over time.
\end{itemize}

Since the objective of this project is to predict a continuous numerical value representing land price, the most suitable approach is supervised regression learning.

\subsection{Regression in Machine Learning}
Regression techniques are used when the target variable is continuous rather than categorical. In this project, regression is employed to estimate land prices based on multiple influential features, including:
\begin{itemize}[leftmargin=*]
    \item Geographic location
    \item Land area
    \item Administrative and political classification
    \item Availability of services and infrastructure
    \item Proximity to main roads and essential facilities
\end{itemize}

Using regression enables objective, consistent, and data-driven valuation while reducing reliance on subjective human estimation, which may vary among assessors.

\subsection{Regression Tree}
A Regression Tree is one of the widely used supervised learning algorithms for predicting continuous values such as real estate and land prices. A regression tree consists of internal nodes, branches, and terminal leaf nodes.

The dataset is recursively divided into increasingly homogeneous subsets by selecting the most informative feature and an appropriate splitting threshold at each node. This process continues until specific stopping criteria are satisfied, such as:
\begin{itemize}[leftmargin=*]
    \item Reaching a maximum tree depth
    \item Reaching a minimum number of samples in a node
    \item Achieving an acceptable prediction error
\end{itemize}

Ultimately, each leaf node represents a group of lands sharing similar characteristics, and a corresponding price estimation is assigned to that group.

\subsection{Overfitting and Pruning}
Regression trees may suffer from overfitting when the tree becomes too complex, learns noise from the training data, and performs poorly on new data. To address this limitation, pruning techniques are applied.

Pre-pruning restricts tree growth through constraints such as limiting maximum depth or requiring a minimum number of samples per node. Post-pruning builds a full tree and then removes branches that do not contribute significantly to prediction performance. These methods enhance the model’s generalization capability and improve the stability of predictions.

\subsection{Reason for Choosing Regression Tree}
The Regression Tree algorithm was selected for this project because it provides interpretable decision-making, supports both numerical and categorical data, performs effectively with small to medium-sized datasets, captures non-linear relationships, and matches the multi-factor nature of land price estimation. Therefore, it represents a scientifically justified and practically efficient choice for predicting land prices in Bani Na’im.

\section{Mathematical Background}
The Regression Tree model implemented in this project is mathematically supported by several fundamental principles governing node splitting, prediction generation, and model complexity control.

\subsection{Sum of Squared Residuals (SSR)}
At each node, the quality of the grouping is evaluated using the Sum of Squared Residuals (SSR), which measures how close the values are to their mean:
\[
SSR = \sum_{i=1}^{n}(y_i - \bar{y})^2
\]
where \(y_i\) is the actual price of sample \(i\), \(\bar{y}\) is the mean price of samples in the node, and \(n\) is the number of samples in the node.
A lower SSR indicates better homogeneity and therefore a better-quality node.

\subsection{Best Split Criterion}
For each potential split, SSR is computed for the left and right subsets. The total resulting error is:
\[
SSR_{total} = SSR_{left} + SSR_{right}
\]
The optimal split is the one that minimizes \(SSR_{total}\), ensuring that the resulting subsets are more homogeneous and stable.

\subsection{Leaf Node Prediction}
After the splitting process terminates, each leaf node represents a set of similar samples. The predicted value assigned to a leaf node is the mean value of all samples within it:
\[
\hat{y} = \frac{1}{n}\sum_{i=1}^{n} y_i
\]
Any new instance that reaches this leaf will be assigned this value as its predicted land price.

\subsection{Model Complexity Control (Cost Complexity Pruning)}
To avoid overfitting, a penalty term is introduced to balance accuracy and structural complexity:
\[
R_\alpha(T) = R(T) + \alpha |T|
\]
where \(R(T)\) is the prediction error of the tree, \(|T|\) is the number of terminal nodes, and \(\alpha\) is a regularization parameter controlling complexity.
Increasing \(\alpha\) reduces tree size and enhances generalization capability.

\subsection{Summary}
These mathematical foundations ensure objective and optimal splitting decisions, accurate prediction generation, and an appropriate balance between accuracy and complexity. As a result, the Regression Tree model used in this project provides reliable, stable, and data-driven land price estimation.

\section{Alternatives}
There are several alternative methods for evaluating the land prices, but the most common method used in Bani Na’im is \textbf{comparative market analysis (CMA)}, which is comparing the land to be evaluated to similar lands that were recently sold. These lands have shared attributes to be compared.

Although this method is commonly used, it is less accurate and less effective than the machine learning method, and also more complicated to justify the result because the CMA method relies heavily on the subjective judgment of experts rather than objective land statistics, which can introduce bias and inconsistency.

In contrast, this project leverages machine learning models, which can automatically learn from data and adapt to changing market and political conditions to provide faster, more accurate predictions and justifiable, transparent results.

\section{Chosen Algorithm for the Model}
We adopt the decision tree regression algorithm as the base machine learning model in this project. This choice is consistent with the theoretical and mathematical background presented in the previous sections and is motivated by the following reasons:

\begin{enumerate}[leftmargin=*]
    \item It can handle both numerical and categorical data.
    \item It works well with non-linear relationships between features and the target.
    \item It provides a clear, interpretable visualization of the decision-making process.
    \item It performs well with small to medium-sized datasets, which matches the available data for Bani Na’im.
    \item It is relatively simple to implement and easy to understand, which facilitates further development and maintenance.
\end{enumerate}

\subsection{Practical Behavior of Decision Tree Regression in This Project}
In the context of this project, the decision tree regression model operates as follows:

\begin{enumerate}[leftmargin=*]
    \item The input features (such as area, location, political classification, and infrastructure availability) are used to recursively split the data into smaller regions.
    \item At each split, the model selects the feature and threshold that minimize the sum of squared residuals, as described in the mathematical background.
    \item Splitting continues until one of the stopping criteria is met, such as maximum depth or minimum number of samples per leaf.
    \item Each terminal node (leaf) stores the average land price of the training samples that fall into that region.
    \item When a new land instance is entered into the system, it is routed through the tree according to its features until it reaches a leaf, where the stored average price is returned as the predicted land value.
\end{enumerate}

This behavior allows the model to capture complex relationships between land characteristics and price, while maintaining a structure that is transparent and explainable to domain experts such as land appraisers.



% -------------------- Chapter 2 --------------------
\chaptertitlepage{Requirement Specifications}

\section{Overview}
This chapter identifies the main users of the land pricing system and describes the role of each. 
It also outlines the functional and non-functional requirements that define how the system should behave, 
and shows how the components of the system interact with each other. 
It also presents visual representations such as a use-case diagram and a context diagram as well as functional requirements tables.


\section{Actors}
The system has three main actors, each with distinct responsibilities:

\begin{enumerate}[leftmargin=*, label=\arabic*-, align=left]
  \item \textbf{Land Appraiser} — Enters target-land characteristics and receives an automated price prediction. Uses the result to validate their own estimate or as a data-backed estimation.
  
  \item \textbf{Admin} — Manages user accounts and system configuration (view roles/emails, activate/deactivate, update or remove users). Maintains a safe, secure, and smooth operation of the platform.
  
  \item \textbf{Data Scientist} — Ensures model and platform quality. Prepares/curates datasets, tests and validates the ML model with real or synthetic data, monitors accuracy and performance, and suggests improvements.
\end{enumerate}

Together, these actors keep the system reliable and continuously improving.

% -------- Context Diagram (صفحة لوحدها + العنوان فوق الصورة) --------
\clearpage
\section{Context Diagram}

\noindent\justifying
Figure 2.3.1 illustrates the context diagram of the AI Land Price Estimation System, showing the main external entities and their interactions with the system. The key entities are the System Admin, Land Appraiser, Data Scientist, and Authentication/Email Service. Each entity communicates with the system through specific commands, data inputs, or reports, ensuring the overall functionality of user management, model development, account security, and land price estimation.

\begin{figure}[H]
  \centering
  \caption{Figure 2.3.1 Context Diagram}
  \vspace{6pt}
  \includegraphics[width=\textwidth,height=1\textheight,keepaspectratio]{images/Context diagram.png}
  \label{fig:context-diagram}
\end{figure}
\clearpage


\section{Functional Requirements}

\subsection{Land Appraiser's Side}
\begin{enumerate}[leftmargin=*,label=\arabic*.,align=left]
  \item \textbf{User Registration and Login}
  \begin{itemize}[leftmargin=1.2em]
    \item Appraisers must be able to register using a valid email address and password.
    \item An activation code provided by the administrator is required to complete registration.
    \item Once registered, appraisers can log in securely using their email and password.
    \item A password reset option must be available in case appraisers forget their password.
  \end{itemize}

  \item \textbf{Profile Management}
  \begin{itemize}[leftmargin=1.2em]
    \item View and edit personal information (e.g., name, email).
    \item Change password from profile settings.
  \end{itemize}

  \item \textbf{Add a New Project}
  \begin{itemize}[leftmargin=1.2em]
    \item Create a new project.
    \item Input land details for estimation.
  \end{itemize}

  \item \textbf{Selecting an Old Project}
  \begin{itemize}[leftmargin=1.2em]
    \item Select a previously created project.
    \item Edit the input data and re-estimate the price.
  \end{itemize}

  \item \textbf{Price Estimation}
  \begin{itemize}[leftmargin=1.2em]
    \item The system processes the entered data and displays the estimated land price.
    \item The appraiser receives a summary of the estimation and the influencing factors.
  \end{itemize}

  \item \textbf{Project History}
  \begin{itemize}[leftmargin=1.2em]
    \item The system saves each submitted land estimation as a separate project.
    \item The appraiser can view a list of all past projects.
    \item Each project shows input details, results, and the date of submission.
  \end{itemize}

  \item \textbf{Edit or Delete Land Inputs (Before Submission)}
  \begin{itemize}[leftmargin=1.2em]
    \item Edit or clear the form data before submitting for estimation.
  \end{itemize}

  \item \textbf{Input Validation}
  \begin{itemize}[leftmargin=1.2em]
    \item The system checks for missing or invalid entries and shows helpful error messages.
  \end{itemize}

  \item \textbf{Rating the Estimation Result}
  \begin{itemize}[leftmargin=1.2em]
    \item The appraiser can rate the estimation result after it is displayed.
  \end{itemize}
\end{enumerate}


\subsection{Admin's Side}
\begin{enumerate}[leftmargin=*,label=\arabic*.,align=left]
  \item \textbf{Login} — The admin can securely log in to the system using their credentials.

  \item \textbf{Manage Users}
  \begin{itemize}[leftmargin=1.2em]
    \item View all registered users.
    \item Remove user accounts.
    \item Edit user roles.
    \item Activate / Deactivate accounts.
  \end{itemize}

  \item \textbf{Creating Admin Accounts} — Create new admin accounts when necessary to expand system management.

  \item \textbf{Manage Form Data} — Manage selectable regions and update system data fields relevant to land evaluation to keep the platform consistent with current geographic and regulatory information.

  \item \textbf{View System Logs} — See records of user activity and system events to monitor and diagnose issues.

  \item \textbf{Manage Backups} — Save backups of system data and restore them in case of data loss or system problems.
\end{enumerate}


\subsection{Data Scientist's Side}
\begin{enumerate}[leftmargin=*,label=\arabic*.,align=left]
  \item \textbf{User Registration and Login}
  \begin{itemize}[leftmargin=1.2em]
    \item Register using a valid email address and password.
    \item Provide an activation code issued by the administrator to complete registration.
    \item Log in securely using credentials once registered.
    \item Reset password when needed.
  \end{itemize}

  \item \textbf{Test Model Accuracy} — Run tests using known or sample land data to evaluate model accuracy.

  \item \textbf{Review Feature Impact} — View which land features (e.g., area, location) most influence the predicted price based on the model’s analysis.

  \item \textbf{Monitor Model Performance Over Time} — Track model performance across time and compare older versions with newer ones.

  \item \textbf{Select Any Project to Analyze} — Select any existing project in the system (including those created by any land appraiser) for analysis.
\end{enumerate}

% -------- Nonfunctional Requirements --------
\section{Nonfunctional Requirements}
The nonfunctional requirements describe how the system should behave to provide the best user experience.

\begin{enumerate}[leftmargin=*,label=\arabic*.,align=left]

  \item \textbf{Usability}
  \begin{itemize}[leftmargin=1.2em]
    \item The system should provide a simple and user-friendly interface.
    \item The interface should support both desktop and mobile browsers.
  \end{itemize}

  \item \textbf{Performance}
  \begin{itemize}[leftmargin=1.2em]
    \item The system should return land price estimation results in less than 5 seconds after submission.
    \item Login and registration should complete in less than 3 seconds under normal load.
  \end{itemize}

  \item \textbf{Availability}
  \begin{itemize}[leftmargin=1.2em]
    \item The system should be available at least 99\% of the time.
  \end{itemize}

  \item \textbf{Security}
  \begin{itemize}[leftmargin=1.2em]
    \item The system must protect user information by applying strong encryption methods.
    \item Passwords should be securely hashed.
    \item Only authorized users can access their personal projects and information.
  \end{itemize}

  \item \textbf{Data Backup and Recovery}
  \begin{itemize}[leftmargin=1.2em]
    \item All user accounts and project details should be backed up regularly.
    \item When a system failure occurs, users should be able to recover their information without data loss.
  \end{itemize}

\end{enumerate}

% -------- Use-Case Diagram (صفحة لوحدها + العنوان فوق الصورة) --------
\clearpage
\section{Use-Case Diagram}
\begin{figure}[H]
  \centering
  \caption{Figure 2.5.1 Use Case Diagram}
  \vspace{6pt}
  \includegraphics[width=\textwidth,height=0.8\textheight,keepaspectratio]{images/use case diagram.png}
  \label{fig:use-case-diagram}
\end{figure}
\clearpage

\section{Appraiser’s Functional Requirements Tables}

% ---- Example Requirement Table Template (copy for other requirements) ----
\begin{table}[H]
\centering
\caption{Table 2.7.1 Login}
\label{tab:req-login}
\begin{tabularx}{\textwidth}{|p{3cm}|X|}
\hline
\textbf{Field} & \textbf{Content} \\ \hline

Requirement & Login \\ \hline

Actor & Land Appraiser \\ \hline

Objective & Access the appraiser’s account \\ \hline

Precondition & The appraiser must be registered. \\ \hline

Scenario & 
\begin{minipage}[t]{0.75\textwidth}
\begin{enumerate}[leftmargin=*,label=\arabic*.]
\item The appraiser enters email and password.
\item The appraiser clicks `Submit`.
\item The system verifies credentials and grants access.
\end{enumerate}
\end{minipage} \\ \hline

Exceptions & 
\begin{enumerate}[leftmargin=*,label=\arabic*.]
\item Incorrect credentials — the system displays an error message.
\item Account locked due to failed attempts.
\item Account not activated.
\item No internet connection.
\item Server or network error — system prompts the user to try again later.
\end{enumerate}
%\end{minipage} 
\\ \hline
\end{tabularx}
\end{table}

% ---- Table 2.7.2 Register Account ----
\begin{table}[H]
\centering
\caption{Table 2.7.2 Register Account}
\label{tab:req-register}
\begin{tabularx}{\textwidth}{|p{3cm}|X|}
\hline
\textbf{Field} & \textbf{Content} \\ \hline

Requirement & Register Account \\ \hline

Actor & Land Appraiser \\ \hline

Objective & Create a new appraiser account. \\ \hline

Precondition & The appraiser must have the activation key from the administrator. \\ \hline

Scenario & 
\begin{minipage}[t]{0.75\textwidth}
\begin{enumerate}[leftmargin=*,label=\arabic*.]
\item The appraiser selects `Register`.
\item The appraiser enters the activation code.
\item The appraiser fills in required details (name, email, phone, password).
\item Verification code is sent to the entered email.
\item The appraiser clicks the link in the email to confirm the email.
\item The appraiser submits the form.
\item The system creates the account and confirms registration.
\end{enumerate}
\end{minipage} \\ \hline

Exceptions & 
\begin{minipage}[t]{0.75\textwidth}
\begin{enumerate}[leftmargin=*,label=\arabic*.]
\item Email already in use.
\item Weak or invalid password.
\item Required fields missing.
\item Activation code expired or incorrect.
\item Server or network error.
\end{enumerate}
\end{minipage} \\ \hline
\end{tabularx}
\end{table}

% ---- Table 2.7.3 Logout ----
\begin{table}[H]
\centering
\caption{Table 2.7.3 Logout}
\label{tab:req-logout}
\begin{tabularx}{\textwidth}{|p{3cm}|X|}
\hline
\textbf{Field} & \textbf{Content} \\ \hline

Requirement & Logout \\ \hline

Actor & Land Appraiser \\ \hline

Objective & Securely end the current session and prevent unauthorized access to the account. \\ \hline

Precondition & The appraiser is logged into the system. \\ \hline

Scenario & 
\begin{minipage}[t]{0.75\textwidth}
\begin{enumerate}[leftmargin=*,label=\arabic*.]
\item The appraiser clicks the Logout button from the system interface.
\item The system ends the current session.
\item The appraiser is sent to the login page.
\end{enumerate}
\end{minipage} \\ \hline

Exceptions & 
\begin{minipage}[t]{0.75\textwidth}
\begin{enumerate}[leftmargin=*,label=\arabic*.]
\item Server or network error.
\end{enumerate}
\end{minipage} \\ \hline
\end{tabularx}
\end{table}

% ---- Table 2.7.4 Reset Password ----
\begin{table}[H]
\centering
\caption{Table 2.7.4 Reset Password}
\label{tab:req-reset-password}
\begin{tabularx}{\textwidth}{|p{3cm}|X|}
\hline
\textbf{Field} & \textbf{Content} \\ \hline

Requirement & Reset Password \\ \hline

Actor & Land Appraiser \\ \hline

Objective & Change the password. \\ \hline

Precondition & Appraiser has a valid registered email. \\ \hline

Scenario & 
\begin{minipage}[t]{0.75\textwidth}
\begin{enumerate}[leftmargin=*,label=\arabic*.]
\item Appraiser clicks "Forgot Password" on the login page or selects "Change Password" from their profile.
\item If "Forgot Password":
  \begin{enumerate}[label*=\arabic*.]
  \item System prompts for the registered email.
  \item Appraiser enters email and submits.
  \item System sends a password reset link or code to the email.
  \item Appraiser clicks the link or enters the code, then sets a new password.
  \end{enumerate}
\item If "Change Password" from profile:
  \begin{enumerate}[label*=\arabic*.]
  \item Appraiser enters current password and new password.
  \item System verifies the current password and updates it.
  \end{enumerate}
\item System confirms that the password has been successfully updated.
\end{enumerate}
\end{minipage} \\ \hline

Exceptions & 
\begin{minipage}[t]{0.75\textwidth}
\begin{enumerate}[leftmargin=*,label=\arabic*.]
\item Email not found in the system.
\item Invalid or expired reset link/code.
\item Incorrect current password (when changing from profile).
\item Server or database error during update.
\end{enumerate}
\end{minipage} \\ \hline
\end{tabularx}
\end{table}

% ---- Table 2.7.5 Create Project ----
\begin{table}[H]
\centering
\caption{Table 2.7.5 Create Project}
\label{tab:req-create-project}
\begin{tabularx}{\textwidth}{|p{3cm}|X|}
\hline
\textbf{Field} & \textbf{Content} \\ \hline

Requirement & Add Project \\ \hline

Actor & Land Appraiser \\ \hline

Objective & Create a new project and enter information needed to estimate the land price. \\ \hline

Precondition & Appraiser must be logged in. \\ \hline

Scenario & 
\begin{minipage}[t]{0.75\textwidth}
\begin{enumerate}[leftmargin=*,label=\arabic*.]
\item Appraiser selects ‘New Project’.
\item Names the Project.
\item Fills in land details.
\item The system checks and validates the input data.
\end{enumerate}
\end{minipage} \\ \hline

Exceptions & 
\begin{minipage}[t]{0.75\textwidth}
\begin{enumerate}[leftmargin=*,label=\arabic*.]
\item Missing or invalid fields — display helpful error messages.
\item Network failure.
\end{enumerate}
\end{minipage} \\ \hline
\end{tabularx}
\end{table}

% ---- Table 2.7.6 Estimate Price ----
\begin{table}[H]
\centering
\caption{Table 2.7.6 Estimate Price}
\label{tab:req-estimate-price}
\begin{tabularx}{\textwidth}{|p{3cm}|X|}
\hline
\textbf{Field} & \textbf{Content} \\ \hline

Requirement & Estimate Price \\ \hline

Actor & Land Appraiser \\ \hline

Objective & Predict and view the price of the land. \\ \hline

Precondition & Land data has been successfully submitted. \\ \hline

Scenario & 
\begin{minipage}[t]{0.75\textwidth}
\begin{enumerate}[leftmargin=*,label=\arabic*.]
\item Appraiser clicks “estimate price”.
\item System runs the model on the input.
\item Displays the estimated price and summary.
\end{enumerate}
\end{minipage} \\ \hline

Exceptions & 
\begin{minipage}[t]{0.75\textwidth}
\begin{enumerate}[leftmargin=*,label=\arabic*.]
\item System error in model execution.
\item Timeout or delay in result.
\item Server or network error.
\end{enumerate}
\end{minipage} \\ \hline
\end{tabularx}
\end{table}

% ---- Table 2.7.7 View Projects ----
\begin{table}[H]
\centering
\caption{Table 2.7.7 View Projects}
\label{tab:req-view-projects}
\begin{tabularx}{\textwidth}{|p{3cm}|X|}
\hline
\textbf{Field} & \textbf{Content} \\ \hline

Requirement & View Projects \\ \hline

Actor & Land Appraiser \\ \hline

Objective & Access previously estimated land projects to view them. \\ \hline

Precondition & Appraiser must be logged in. \\ \hline

Scenario & 
\begin{minipage}[t]{0.75\textwidth}
\begin{enumerate}[leftmargin=*,label=\arabic*.]
\item Appraiser selects ‘My Projects’.
\item System displays a list of past projects.
\item Appraiser can select any project to view.
\end{enumerate}
\end{minipage} \\ \hline

Exceptions & 
\begin{minipage}[t]{0.75\textwidth}
\begin{enumerate}[leftmargin=*,label=\arabic*.]
\item No saved projects.
\item Database access failure.
\item Network error.
\end{enumerate}
\end{minipage} \\ \hline
\end{tabularx}
\end{table}

% ---- Table 2.7.8 Update Projects ----
\begin{table}[H]
\centering
\caption{Table 2.7.8 Update Projects}
\label{tab:req-update-projects}
\begin{tabularx}{\textwidth}{|p{3cm}|X|}
\hline
\textbf{Field} & \textbf{Content} \\ \hline

Requirement & Update Projects \\ \hline

Actor & Land Appraiser \\ \hline

Objective & Edit previously estimated land projects, perform new estimations, and keep a record of past estimations for the same project. \\ \hline

Precondition & Appraiser must be logged in. \\ \hline

Scenario & 
\begin{minipage}[t]{0.75\textwidth}
\begin{enumerate}[leftmargin=*,label=\arabic*.]
\item Appraiser selects ‘My Projects’.
\item System displays a list of past projects.
\item Appraiser selects a project to update.
\item Appraiser edits the project details.
\item System generates a new estimation for the updated data.
\item Previous estimations for the same project are saved and can be viewed.
\end{enumerate}
\end{minipage} \\ \hline

Exceptions & 
\begin{minipage}[t]{0.75\textwidth}
\begin{enumerate}[leftmargin=*,label=\arabic*.]
\item No saved projects.
\item Database access failure.
\item Network error.
\end{enumerate}
\end{minipage} \\ \hline
\end{tabularx}
\end{table}

% ---- Table 2.7.9 View Profile ----
\begin{table}[H]
\centering
\caption{Table 2.7.9 View Profile}
\label{tab:req-view-profile}
\begin{tabularx}{\textwidth}{|p{3cm}|X|}
\hline
\textbf{Field} & \textbf{Content} \\ \hline

Requirement & View Profile \\ \hline

Actor & Land Appraiser \\ \hline

Objective & View the appraiser’s account information. \\ \hline

Precondition & Appraiser must be logged in. \\ \hline

Scenario & 
\begin{minipage}[t]{0.75\textwidth}
\begin{enumerate}[leftmargin=*,label=\arabic*.]
\item Appraiser selects ‘Profile’.
\item System displays current account information, including name, email, phone, and other profile details.
\end{enumerate}
\end{minipage} \\ \hline

Exceptions & 
\begin{minipage}[t]{0.75\textwidth}
\begin{enumerate}[leftmargin=*,label=\arabic*.]
\item Profile data not found.
\item Database access failure.
\item Network error.
\end{enumerate}
\end{minipage} \\ \hline
\end{tabularx}
\end{table}

% ---- Table 2.7.10 Edit Profile ----
\begin{table}[H]
\centering
\caption{Table 2.7.10 Edit Profile}
\label{tab:req-edit-profile}
\begin{tabularx}{\textwidth}{|p{3cm}|X|}
\hline
\textbf{Field} & \textbf{Content} \\ \hline

Requirement & Edit Profile \\ \hline

Actor & Land Appraiser \\ \hline

Objective & Update appraiser’s account information. \\ \hline

Precondition & Appraiser must be logged in and in the profile. \\ \hline

Scenario & 
\begin{minipage}[t]{0.75\textwidth}
\begin{enumerate}[leftmargin=*,label=\arabic*.]
\item Appraiser selects ‘Edit Profile’.
\item Updates name, email, phone number, or password.
\item Clicks ‘Save Changes’.
\end{enumerate}
\end{minipage} \\ \hline

Exceptions & 
\begin{minipage}[t]{0.75\textwidth}
\begin{enumerate}[leftmargin=*,label=\arabic*.]
\item Invalid email or password.
\item Server error during update.
\item Network or server error.
\end{enumerate}
\end{minipage} \\ \hline
\end{tabularx}
\end{table}

% ---- Table 2.7.11 Rate Estimation Result ----
\begin{table}[H]
\centering
\caption{Table 2.7.11 Rate Estimation Result}
\label{tab:req-rate-estimation}
\begin{tabularx}{\textwidth}{|p{3cm}|X|}
\hline
\textbf{Field} & \textbf{Content} \\ \hline

Requirement & Rate Estimation Result \\ \hline

Actor & Land Appraiser \\ \hline

Objective & Provide feedback on the quality of the estimated result to help evaluate model performance. \\ \hline

Precondition & An estimation result must be displayed. \\ \hline

Scenario & 
\begin{minipage}[t]{0.75\textwidth}
\begin{enumerate}[leftmargin=*,label=\arabic*.]
\item After viewing the estimation result, the appraiser is prompted to provide feedback.
\item The appraiser selects a rating option (e.g., Thumbs Up or Thumbs Down).
\item The appraiser can provide a more logical/accurate estimation in case of negative rating.
\item The system saves the rating along with the estimation details and corrected estimation for the current project.
\end{enumerate}
\end{minipage} \\ \hline

Exceptions & 
\begin{minipage}[t]{0.75\textwidth}
\begin{enumerate}[leftmargin=*,label=\arabic*.]
\item Rating submission fails due to network error.
\item Database save error.
\end{enumerate}
\end{minipage} \\ \hline
\end{tabularx}
\end{table}



\section{Admin’s Functional Requirements Tables}
% ---- Table 2.8.1 Login ----
\begin{table}[H]
\centering
\caption{Table 2.8.1 Login}
\label{tab:admin-login}
\begin{tabularx}{\textwidth}{|p{3cm}|X|}
\hline
\textbf{Field} & \textbf{Content} \\ \hline

Requirement & Login \\ \hline

Actor & Admin \\ \hline

Objective & Allow admin to securely log in to the system. \\ \hline

Precondition & Admin must be registered and approved as an administrator by an existing admin. \\ \hline

Scenario & 
\begin{minipage}[t]{0.75\textwidth}
\begin{enumerate}[leftmargin=*,label=\arabic*.]
\item Admin visits login page.
\item Enters email and password.
\item Clicks "Login".
\item The system verifies credentials and grants access.
\end{enumerate}
\end{minipage} \\ \hline

Exceptions &
\begin{minipage}[t]{0.75\textwidth}
\begin{enumerate}[leftmargin=*,label=\arabic*.]
\item Invalid email or password — system displays an error message.
\item Server or network error — system prompts the user to try again later.
\end{enumerate}
\end{minipage} \\ \hline

\end{tabularx}
\end{table}

% ---- Table 2.8.2 Logout ----
\begin{table}[H]
\centering
\caption{Table 2.8.2 Logout}
\label{tab:admin-logout}
\begin{tabularx}{\textwidth}{|p{3cm}|X|}
\hline
\textbf{Field} & \textbf{Content} \\ \hline

Requirement & Logout \\ \hline

Actor & Admin \\ \hline

Objective & Securely end the current session and prevent unauthorized access to the account. \\ \hline

Precondition & Admin is logged into the system. \\ \hline

Scenario & 
\begin{minipage}[t]{0.75\textwidth}
\begin{enumerate}[leftmargin=*,label=\arabic*.]
\item Admin clicks the Logout button from the system interface.
\item The system ends the current session.
\item Admin is sent to the login page.
\end{enumerate}
\end{minipage} \\ \hline

Exceptions &
\begin{minipage}[t]{0.75\textwidth}
\begin{enumerate}[leftmargin=*,label=\arabic*.]
\item Server or network error.
\end{enumerate}
\end{minipage} \\ \hline

\end{tabularx}
\end{table}

% ---- Table 2.8.3 Creating Admin Account ----
\begin{table}[H]
\centering
\caption{Table 2.8.3 Creating Admin Account}
\label{tab:admin-create-account}
\begin{tabularx}{\textwidth}{|p{3cm}|X|}
\hline
\textbf{Field} & \textbf{Content} \\ \hline

Requirement & Creating Admin Accounts \\ \hline

Actor & Existing Admin \\ \hline

Objective & Create new admin accounts. \\ \hline

Precondition & Existing Admin is authorized to create admins. \\ \hline

Scenario & 
\begin{minipage}[t]{0.75\textwidth}
\begin{enumerate}[leftmargin=*,label=\arabic*.]
\item Authorized admin accesses the admin management interface.
\item Creates an admin and enters admin details.
\item Account is activated.
\end{enumerate}
\end{minipage} \\ \hline

Exceptions &
\begin{minipage}[t]{0.75\textwidth}
\begin{enumerate}[leftmargin=*,label=\arabic*.]
\item Invitation link expires.
\item Unauthorized requester attempts to create an admin.
\item Failure in account setup due to system error.
\end{enumerate}
\end{minipage} \\ \hline

\end{tabularx}
\end{table}

% ---- Table 2.8.4 Manage Users ----
\begin{table}[H]
\centering
\caption{Table 2.8.4 Manage Users}
\label{tab:admin-manage-users}
\begin{tabularx}{\textwidth}{|p{3cm}|X|}
\hline
\textbf{Field} & \textbf{Content} \\ \hline

Requirement & View Users \\ \hline

Actor & Admin \\ \hline

Objective & View a list of all registered users with their details, and the ability to select any user to edit their account. \\ \hline

Precondition & Admin is logged in. \\ \hline

Scenario & 
\begin{minipage}[t]{0.75\textwidth}
\begin{enumerate}[leftmargin=*,label=\arabic*.]
\item Admin opens the user management panel.
\item System displays a list of all registered users with basic details (e.g., name, email, registration date, role).
\item Admin can sort or filter the list.
\item Admin can select any user to make actions.
\item The actions are: Delete User, Deactivate Account (if activated), Activate Account (if deactivated), and Change Role.
\end{enumerate}
\end{minipage} \\ \hline

Exceptions &
\begin{minipage}[t]{0.75\textwidth}
\begin{enumerate}[leftmargin=*,label=\arabic*.]
\item No users found in the system.
\item Server error when retrieving user data.
\item Database connection failure.
\item Failure of action.
\end{enumerate}
\end{minipage} \\ \hline

\end{tabularx}
\end{table}


% ---- Table 2.8.5 Manage Form Data ----
\begin{table}[H]
\centering
\caption{Table 2.8.5 Manage Form Data}
\label{tab:admin-manage-form-data}
\begin{tabularx}{\textwidth}{|p{3cm}|X|}
\hline
\textbf{Field} & \textbf{Content} \\ \hline

Requirement & Manage Form Data \\ \hline

Actor & Admin \\ \hline

Objective & Edit or add options (e.g., locations or classifications) available during project creation. \\ \hline

Precondition & Admin has access rights. \\ \hline

Scenario & 
\begin{minipage}[t]{0.75\textwidth}
\begin{enumerate}[leftmargin=*,label=\arabic*.]
\item Admin selects ‘Manage Data’.
\item Chooses data category (e.g., regions).
\item Edits, adds, or deletes entries.
\item Saves changes.
\end{enumerate}
\end{minipage} \\ \hline

Exceptions &
\begin{minipage}[t]{0.75\textwidth}
\begin{enumerate}[leftmargin=*,label=\arabic*.]
\item Input is invalid.
\item Changes not saved due to a database error.
\end{enumerate}
\end{minipage} \\ \hline

\end{tabularx}
\end{table}

% ---- Table 2.8.6 View System Logs ----
\begin{table}[H]
\centering
\caption{Table 2.8.6 View System Logs}
\label{tab:admin-view-logs}
\begin{tabularx}{\textwidth}{|p{3cm}|X|}
\hline
\textbf{Field} & \textbf{Content} \\ \hline

Requirement & View System Logs \\ \hline

Actor & Admin \\ \hline

Objective & Monitor system events and user activity. \\ \hline

Precondition & System logging is enabled. \\ \hline

Scenario &
\begin{minipage}[t]{0.75\textwidth}
\begin{enumerate}[leftmargin=*,label=\arabic*.]
\item Admin navigates to the ‘Logs’ section.
\item Filters by date or activity type.
\item Views login, registration, or error logs.
\end{enumerate}
\end{minipage} \\ \hline

Exceptions &
\begin{minipage}[t]{0.75\textwidth}
\begin{enumerate}[leftmargin=*,label=\arabic*.]
\item Logs not available.
\item Permission denied.
\end{enumerate}
\end{minipage} \\ \hline

\end{tabularx}
\end{table}

% ---- Table 2.8.7 Manage Backups ----
\begin{table}[H]
\centering
\caption{Table 2.8.7 Manage Backups}
\label{tab:admin-manage-backups}
\begin{tabularx}{\textwidth}{|p{3cm}|X|}
\hline
\textbf{Field} & \textbf{Content} \\ \hline

Requirement & Manage Backups \\ \hline

Actor & Admin \\ \hline

Objective & Ensure system and user data is regularly backed up. \\ \hline

Precondition & Backup system is active. \\ \hline

Scenario &
\begin{minipage}[t]{0.75\textwidth}
\begin{enumerate}[leftmargin=*,label=\arabic*.]
\item Admin opens ‘Backup Settings’.
\item Triggers manual backup or sets automatic schedule.
\item Confirms successful completion.
\end{enumerate}
\end{minipage} \\ \hline

Exceptions &
\begin{minipage}[t]{0.75\textwidth}
\begin{enumerate}[leftmargin=*,label=\arabic*.]
\item Backup failed due to storage limit.
\item Scheduled backup skipped due to server downtime.
\end{enumerate}
\end{minipage} \\ \hline

\end{tabularx}
\end{table}

% ---- Table 2.8.8 Create Activation Key ----
\begin{table}[H]
\centering
\caption{Table 2.8.8 Create Activation Key}
\label{tab:admin-create-key}
\begin{tabularx}{\textwidth}{|p{3cm}|X|}
\hline
\textbf{Field} & \textbf{Content} \\ \hline

Requirement & Create Activation Key \\ \hline

Actor & Admin \\ \hline

Objective & Generate a unique activation key for a data scientist or appraiser to use when registering their account. \\ \hline

Precondition & Admin is logged in. \\ \hline

Scenario &
\begin{minipage}[t]{0.75\textwidth}
\begin{enumerate}[leftmargin=*,label=\arabic*.]
\item Admin opens the "Activation Keys" panel.
\item Selects the account type (Data Scientist or Appraiser).
\item Clicks "Generate Key".
\item System generates a unique activation key.Clicks
\item Admin copies or sends the key Clicksto the intended recipient.
\end{enumerate}
\end{minipage} \\ \hline
Exceptions &
\begin{minipage}[t]{0.75\textwidth}
\begin{enumerate}[leftmargin=*,label=\arabic*.]
\item Server error during key generation.
\item Database access failure when saving the new key.

\end{enumerate}
\end{minipage} \\ \hline
\end{tabularx}
\end{table}

% ---- Table 2.8.9 Reset Password ----
\begin{table}[H]
\centering
\caption{Table 2.8.9 Reset Password}
\label{tab:admin-reset-password}
\begin{tabularx}{\textwidth}{|p{3cm}|X|}
\hline
\textbf{Field} & \textbf{Content} \\ \hline

Requirement & Reset Password \\ \hline

Actor & Admin \\ \hline

Objective & Change the password. \\ \hline

Precondition & Admin has a registered account. \\ \hline

Scenario &
\begin{minipage}[t]{0.75\textwidth}
\begin{enumerate}[leftmargin=*,label=\arabic*.]
\item Admin clicks "Forgot Password" on the login page or selects "Change Password" from their profile.
\item If "Forgot Password":
    \begin{enumerate}[label*=\arabic*.]
    \item System prompts for the registered email.
    \item Admin enters email and submits.
    \item System sends a password reset link or code to the email.
    \item Admin clicks the link or enters the code, then sets a new password.
    \end{enumerate}
\item If "Change Password" from profile:
    \begin{enumerate}[label*=\arabic*.]
    \item Admin enters current password and new password.
    \item System verifies the current password and updates it.
    \end{enumerate}
\item System confirms that the password has been successfully updated.
\end{enumerate}
\end{minipage} \\ \hline

Exceptions &
\begin{minipage}[t]{0.75\textwidth}
\begin{enumerate}[leftmargin=*,label=\arabic*.]
\item Email not found in the system.
\item Invalid or expired reset link/code.
\item Incorrect current password (when changing from profile).
\item Server or database error during update.
\end{enumerate}
\end{minipage} \\ \hline

\end{tabularx}
\end{table}

% ---- Table 2.8.10 View Profile ----
\begin{table}[H]
\centering
\caption{Table 2.8.10 View Profile}
\label{tab:admin-view-profile}
\begin{tabularx}{\textwidth}{|p{3cm}|X|}
\hline
\textbf{Field} & \textbf{Content} \\ \hline

Requirement & View Profile \\ \hline

Actor & Admin \\ \hline

Objective & View the admin’s account information. \\ \hline

Precondition & Admin must be logged in. \\ \hline

Scenario &
\begin{minipage}[t]{0.75\textwidth}
\begin{enumerate}[leftmargin=*,label=\arabic*.]
\item Admin selects ‘Profile’.
\item System displays current account information, including name, email, phone, and other profile details.
\end{enumerate}
\end{minipage} \\ \hline

Exceptions &
\begin{minipage}[t]{0.75\textwidth}
\begin{enumerate}[leftmargin=*,label=\arabic*.]
\item Profile data not found.
\item Database access failure.
\item Network error.
\end{enumerate}
\end{minipage} \\ \hline

\end{tabularx}
\end{table}

% ---- Table 2.8.11 Edit Profile ----
\begin{table}[H]
\centering
\caption{Table 2.8.11 Edit Profile}
\label{tab:admin-edit-profile}
\begin{tabularx}{\textwidth}{|p{3cm}|X|}
\hline
\textbf{Field} & \textbf{Content} \\ \hline

Requirement & Edit Profile \\ \hline

Actor & Admin \\ \hline

Objective & Update Admin’s account information. \\ \hline

Precondition & Admin must be logged in and in the profile. \\ \hline

Scenario & 
\begin{minipage}[t]{0.75\textwidth}
\begin{enumerate}[leftmargin=*,label=\arabic*.]
\item Admin selects ‘Edit Profile’.
\item Updates name, email, phone number, or password.
\item Clicks ‘Save Changes’.
\end{enumerate}
\end{minipage} \\ \hline

Exceptions & 
\begin{minipage}[t]{0.75\textwidth}
\begin{enumerate}[leftmargin=*,label=\arabic*.]
\item Invalid email or password.
\item Server error during update.
\item Network or server error.
\end{enumerate}
\end{minipage} \\ \hline

\end{tabularx}
\end{table}


\section{Data Scientist’s Functional Requirements Tables}
% ---- Table 2.9.1 Login ----
\begin{table}[H]
\centering
\caption{Table 2.9.1 Login}
\label{tab:ds-login}
\begin{tabularx}{\textwidth}{|p{3cm}|X|}
\hline
\textbf{Field} & \textbf{Content} \\ \hline

Requirement & Login \\ \hline

Actor & Data Scientist \\ \hline

Objective & Allow the data scientist to securely log in to the system. \\ \hline

Precondition & The data scientist must be registered and approved as a data scientist by an admin. \\ \hline

Scenario & 
\begin{minipage}[t]{0.75\textwidth}
\begin{enumerate}[leftmargin=*,label=\arabic*.]
\item Data scientist visits login page.
\item Enters a valid email and password.
\item Clicks the Login button.
\item The system verifies credentials and grants access.
\end{enumerate}
\end{minipage} \\ \hline

Exceptions & 
\begin{minipage}[t]{0.75\textwidth}
\begin{enumerate}[leftmargin=*,label=\arabic*.]
\item Invalid email or password — system displays an error message.
\item Server or network error — system prompts the user to try again later.
\end{enumerate}
\end{minipage} \\ \hline

\end{tabularx}
\end{table}

% ---- Table 2.9.2 Logout ----
\begin{table}[H]
\centering
\caption{Table 2.9.2 Logout}
\label{tab:ds-logout}
\begin{tabularx}{\textwidth}{|p{3cm}|X|}
\hline
\textbf{Field} & \textbf{Content} \\ \hline

Requirement & Logout \\ \hline

Actor & Data Scientist \\ \hline

Objective & Securely end the current session and prevent unauthorized access to the account. \\ \hline

Precondition & The Data Scientist is logged into the system. \\ \hline

Scenario & 
\begin{minipage}[t]{0.75\textwidth}
\begin{enumerate}[leftmargin=*,label=\arabic*.]
\item The data scientist clicks the Logout button from the system.
\item The system ends the current session.
\item The data scientist is sent to the login page.
\end{enumerate}
\end{minipage} \\ \hline

Exceptions & 
\begin{minipage}[t]{0.75\textwidth}
\begin{enumerate}[leftmargin=*,label=\arabic*.]
\item Server or network error.
\end{enumerate}
\end{minipage} \\ \hline

\end{tabularx}
\end{table}

% ---- Table 2.9.3 Register Account ----
\begin{table}[H]
\centering
\caption{Table 2.9.3 Register Account}
\label{tab:ds-register}
\begin{tabularx}{\textwidth}{|p{3cm}|X|}
\hline
\textbf{Field} & \textbf{Content} \\ \hline

Requirement & Register Account \\ \hline

Actor & Data Scientist \\ \hline

Objective & Create a new Data Scientist account. \\ \hline

Precondition & The Data Scientist must have the activation key from the administrator. \\ \hline

Scenario & 
\begin{minipage}[t]{0.75\textwidth}
\begin{enumerate}[leftmargin=*,label=\arabic*.]
\item The Data Scientist selects ‘Register’.
\item Data Scientist enters the activation code.
\item The Data Scientist fills in required details (name, email, password).
\item Verification code is sent to the entered email.
\item Data Scientist clicks the link in the email to confirm the email.
\item The Data Scientist submits the form.
\item The system creates the account and confirms registration.
\end{enumerate}
\end{minipage} \\ \hline

Exceptions &
\begin{minipage}[t]{0.75\textwidth}
\begin{enumerate}[leftmargin=*,label=\arabic*.]
\item Email already in use.
\item Weak or invalid password.
\item Required fields missing.
\item Activation code expired or incorrect.
\item Server or network error.
\end{enumerate}
\end{minipage} \\ \hline

\end{tabularx}
\end{table}

% ---- Table 2.9.4 Test Model Accuracy ----
\begin{table}[H]
\centering
\caption{Table 2.9.4 Test Model Accuracy}
\label{tab:ds-test-model}
\begin{tabularx}{\textwidth}{|p{3cm}|X|}
\hline
\textbf{Field} & \textbf{Content} \\ \hline

Requirement & Test Model Accuracy \\ \hline

Actor & Data Scientist \\ \hline

Objective & Evaluate the prediction accuracy of the machine learning model. \\ \hline

Precondition & The system must have a trained model and a dataset available for testing. \\ \hline

Scenario &
\begin{minipage}[t]{0.75\textwidth}
\begin{enumerate}[leftmargin=*,label=\arabic*.]
\item The data scientist selects the "Model Testing" section.
\item Uploads or selects a dataset for testing.
\item Runs the model to predict prices.
\item Compares predicted results with actual prices.
\end{enumerate}
\end{minipage} \\ \hline

Exceptions &
\begin{minipage}[t]{0.75\textwidth}
\begin{enumerate}[leftmargin=*,label=\arabic*.]
\item Incomplete or invalid test dataset.
\item Model not available or not trained.
\end{enumerate}
\end{minipage} \\ \hline

\end{tabularx}
\end{table}

% ---- Table 2.9.5 Review Feature Impact ----
\begin{table}[H]
\centering
\caption{Table 2.9.5 Review Feature Impact}
\label{tab:ds-feature-impact}
\begin{tabularx}{\textwidth}{|p{3cm}|X|}
\hline
\textbf{Field} & \textbf{Content} \\ \hline

Requirement & Review Feature Impact \\ \hline

Actor & Data Scientist \\ \hline

Objective & Analyze which features had the most influence on the land price prediction for a specific project. \\ \hline

Precondition & 
\begin{minipage}[t]{0.75\textwidth}
\begin{enumerate}[leftmargin=*,label=\arabic*.]
\item The model must be trained and support feature importance analysis.
\item The project must have completed the estimation.
\end{enumerate}
\end{minipage} \\ \hline

Scenario &
\begin{minipage}[t]{0.75\textwidth}
\begin{enumerate}[leftmargin=*,label=\arabic*.]
\item Access the “Projects” section.
\item Select a specific project from the list.
\item Open the “Feature Importance” view for that project.
\item View ranked list of features by their impact on the project’s prediction.
\item Export or download the report if needed.
\end{enumerate}
\end{minipage} \\ \hline

Exceptions &
\begin{minipage}[t]{0.75\textwidth}
\begin{enumerate}[leftmargin=*,label=\arabic*.]
\item Feature analysis tool is unavailable or unsupported for that project.
\item Insufficient project data to generate meaningful insights.
\item Selected project not found or inaccessible.
\end{enumerate}
\end{minipage} \\ \hline

\end{tabularx}
\end{table}


% ---- Table 2.9.6 Monitor Model Performance Over Time ----
\begin{table}[H]
\centering
\caption{Table 2.9.6 Monitor Model Performance Over Time}
\label{tab:ds-monitor-model}
\begin{tabularx}{\textwidth}{|p{3cm}|X|}
\hline
\textbf{Field} & \textbf{Content} \\ \hline

Requirement & Monitor Model Performance Over Time \\ \hline

Actor & Data Scientist \\ \hline

Objective & Track how the model performs across different versions and datasets, and test any selected version on demand. \\ \hline

Precondition & 
\begin{minipage}[t]{0.75\textwidth}
\begin{enumerate}[leftmargin=*,label=\arabic*.]
\item System must store model versions, related datasets, and performance logs.
\item At least one model version must exist.
\end{enumerate}
\end{minipage} \\ \hline

Scenario & 
\begin{minipage}[t]{0.75\textwidth}
\begin{enumerate}[leftmargin=*,label=\arabic*.]
\item Go to the “Model History” tab.
\item System displays a list of stored model versions with their details.
\item Select a version to see its past results.
\item Optionally, choose a dataset to re-test the selected model version.
\item System runs the test and shows the new results.
\end{enumerate}
\end{minipage} \\ \hline

Exceptions & 
\begin{minipage}[t]{0.75\textwidth}
\begin{enumerate}[leftmargin=*,label=\arabic*.]
\item No stored model versions available.
\item Past performance data is missing or incomplete.
\item Testing fails due to corrupted data or unsupported dataset format.
\item Network or server error during testing.
\end{enumerate}
\end{minipage} \\ \hline

\end{tabularx}
\end{table}

% ---- Table 2.9.7 Select Project to Analyze ----
\begin{table}[H]
\centering
\caption{Table 2.9.7 Select Project to Analyze}
\label{tab:ds-select-project}
\begin{tabularx}{\textwidth}{|p{3cm}|X|}
\hline
\textbf{Field} & \textbf{Content} \\ \hline

Requirement & Select Project to Analyze \\ \hline

Actor & Data Scientist \\ \hline

Objective & Choose a specific project from the list of available projects to perform analysis on its estimations and related data. \\ \hline

Precondition & 
\begin{minipage}[t]{0.75\textwidth}
\begin{enumerate}[leftmargin=*,label=\arabic*.]
\item Data Scientist must be logged in.
\item There must be at least one project available to analyze.
\end{enumerate}
\end{minipage} \\ \hline

Scenario & 
\begin{minipage}[t]{0.75\textwidth}
\begin{enumerate}[leftmargin=*,label=\arabic*.]
\item Data Scientist opens the “Projects” section.
\item System displays a list of projects with basic details (e.g., project name, date, owner).
\item Data Scientist searches, filters, or sorts projects if needed.
\item Selects a project to open for detailed analysis.
\end{enumerate}
\end{minipage} \\ \hline

Exceptions & 
\begin{minipage}[t]{0.75\textwidth}
\begin{enumerate}[leftmargin=*,label=\arabic*.]
\item No available projects to analyze.
\item Project data is incomplete or inaccessible.
\item Network or server error while loading projects.
\end{enumerate}
\end{minipage} \\ \hline

\end{tabularx}
\end{table}

% ---- Table 2.9.8 Reset Password ----
\begin{table}[H]
\centering
\caption{Table 2.9.8 Reset Password}
\label{tab:ds-reset-password}
\begin{tabularx}{\textwidth}{|>{\raggedright\arraybackslash}p{3cm}|X|}
\hline
\textbf{Field} & \textbf{Content} \\ \hline

Requirement & Reset Password \\ \hline
Actor & Data Scientist \\ \hline
Objective & Change the password. \\ \hline
Precondition & The data scientist has a valid registered email. \\ \hline

Scenario &
\begin{minipage}[t]{\linewidth}
\begin{enumerate}[leftmargin=*,label=\arabic*.,itemsep=0pt,topsep=2pt]
  \item The data scientist clicks ``Forgot Password'' on the login page or selects ``Change Password'' from their profile.
  \item If ``Forgot Password'':
    \begin{enumerate}[label*=\arabic*.,itemsep=0pt,topsep=2pt]
      \item System prompts for the registered email.
      \item The data scientist enters the email and submits.
      \item System sends a password reset link or code to the email.
      \item The data scientist clicks the link or enters the code, then sets a new password.
    \end{enumerate}
  \item If ``Change Password'' from profile:
    \begin{enumerate}[label*=\arabic*.,itemsep=0pt,topsep=2pt]
      \item The data scientist enters current password and new password.
      \item System verifies the current password and updates it.
    \end{enumerate}
  \item System confirms that the password has been successfully updated.
\end{enumerate}
\end{minipage}
\\ \hline

Exceptions &
\begin{minipage}[t]{\linewidth}
\begin{enumerate}[leftmargin=*,label=\arabic*.,itemsep=0pt,topsep=2pt]
  \item Email not found in the system.
  \item Invalid or expired reset link/code.
  \item Incorrect current password (when changing from profile).
  \item Server or database error during update.
\end{enumerate}
\end{minipage}
\\ \hline

\end{tabularx}
\end{table}


% ---- Table 2.9.9 View Profile ----
\begin{table}[H]
\centering
\caption{Table 2.9.9 View Profile}
\label{tab:ds-view-profile}
\begin{tabularx}{\textwidth}{|p{3cm}|X|}
\hline
\textbf{Field} & \textbf{Content} \\ \hline

Requirement & View Profile \\ \hline

Actor & Data Scientist \\ \hline

Objective & View the Data Scientist’s account information. \\ \hline

Precondition & Data Scientist must be logged in. \\ \hline

Scenario & 
\begin{minipage}[t]{0.75\textwidth}
\begin{enumerate}[leftmargin=*,label=\arabic*.]
\item Data Scientist selects ‘Profile’.
\item System displays current account information, including name, email, phone, and other profile details.
\end{enumerate}
\end{minipage} \\ \hline

Exceptions & 
\begin{minipage}[t]{0.75\textwidth}
\begin{enumerate}[leftmargin=*,label=\arabic*.]
\item Profile data not found.
\item Database access failure.
\item Network error.
\end{enumerate}
\end{minipage} \\ \hline

\end{tabularx}
\end{table}

% ---- Table 2.9.10 Edit Profile ----
\begin{table}[H]
\centering
\caption{Table 2.9.10 Edit Profile}
\label{tab:ds-edit-profile}
\begin{tabularx}{\textwidth}{|p{3cm}|X|}
\hline
\textbf{Field} & \textbf{Content} \\ \hline

Requirement & Edit Profile \\ \hline

Actor & Data Scientist \\ \hline

Objective & Update Data Scientist’s account information. \\ \hline

Precondition & Data Scientist must be logged in and in the profile. \\ \hline

Scenario & 
\begin{minipage}[t]{0.75\textwidth}
\begin{enumerate}[leftmargin=*,label=\arabic*.]
\item Data Scientist selects ‘Edit Profile’.
\item Updates name, email, phone number, or password.
\item Clicks ‘Save Changes’.
\end{enumerate}
\end{minipage} \\ \hline

Exceptions & 
\begin{minipage}[t]{0.75\textwidth}
\begin{enumerate}[leftmargin=*,label=\arabic*.]
\item Invalid email or password.
\item Server error during update.
\item Network or server error.
\end{enumerate}
\end{minipage} \\ \hline

\end{tabularx}
\end{table}


% -------------------- Chapter 3 --------------------
\chaptertitlepage{Architecture and Design}

\section{Overview}
\noindent\justifying
This chapter explains how the Land Price Estimator system is organized and how its parts work together. It covers the system's design, the chosen architecture and its possible alternatives, the database structure, and the main interfaces for the user, administrator, and data scientist.


\section{Chosen Architecture Design}
\noindent\justifying
We studied multiple architecture design options, and concluded that the MVT (Model–View–Template) architecture is the best fit for our project.

\begin{table}[H]
\centering
\caption{Table 3.2.1 MVT Components}
\label{tab:mvt-components}
\setlength{\arrayrulewidth}{0.6pt}   % سماكة حدود الخلايا (اختياري)
\renewcommand{\arraystretch}{1.25}   % تهوية الصفوف (اختياري)
\begin{tabularx}{\textwidth}{|>{\raggedright\arraybackslash}p{3.2cm}|X|}
\hline
\textbf{Component} & \textbf{Role} \\ \hline
Model    & Manages data, database structure, and rules. \\ \hline
View     & Handles user actions; retrieves data from the model and selects the appropriate template to display results. \\ \hline
Template & Presentation layer controlling how data is rendered to the user (HTML). \\ \hline
\end{tabularx}
\end{table}


The separation makes it easy for developers to work on different components of the application at the same time without affecting each other's work, and makes future scalability as well as maintaining, debugging, and testing the application easier.

\noindent\textbf{Why MVT?} The MVT architecture is provided by Django, which is the framework we are using to develop the web application for the Land-Price Prediction system.


\section{Architecture Implementation}
In our Land-Price Prediction system, the MVT architecture is implemented as follows:

\begin{description}
    \item[Model:] Stores all the data related to each entity of the system, such as the lands and users attributes, and how they are stored in the database and how to retrieve them.
    
    \item[View:] The view stores the business logic and connects the models with the templates; it processes user requests, retrieves data from the model, and gives it to the template.
    
    \item[Template:] It's the interface that the user sees and interacts with. Through it, the user sees the results of the predictions and other information. The template has no business logic to ensure a clean separation from backend processing.
\end{description}

This structured method ensures that each layer is independent but still connected.

\begin{figure}[H]
  \captionsetup{position=top, labelformat=empty, labelsep=none}
  \centering
  \vspace{1cm}
   \caption{Figure 3.3.1 MVT Architecture}
  \includegraphics[width=\textwidth]{images/MVT.png}
  \label{fig:mvt}
\end{figure}



\subsection{Example Models in the System}

The project will have multiple models to manage the data effectively within the MVT architecture. Each model represents an entity of the Land-Price Prediction system and has its own attributes. Example models include:

\begin{enumerate}
    \item \textbf{User:} Represents a system user. Attributes include \texttt{id}, \texttt{full\_name}, \texttt{email}, \texttt{password}, \texttt{role}, and \texttt{created\_at}.
    
    \item \textbf{Project:} Represents a land estimation project. Attributes include \texttt{id}, \texttt{name}, \texttt{description}, \texttt{created\_by}, \texttt{created\_at}, and \texttt{status}.
    
    \item \textbf{Plot (Land Parcel):} Represents a land parcel. Attributes include \texttt{id}, \texttt{plot\_code}, \texttt{governorate\_id}, \texttt{town\_id}, \texttt{neighborhood\_id}, \texttt{area\_m2}, \texttt{slope}, \texttt{soil\_type\_id}, \texttt{rock\_type\_id}, \texttt{current\_land\_use}, \texttt{planned\_land\_use}, \texttt{far}, \texttt{coverage\_ratio}, \texttt{is\_current}, \texttt{version\_no}, \texttt{created\_by}, and \texttt{created\_at}.
    
    \item \textbf{Plot Document:} Represents documents related to a plot. Attributes include \texttt{id}, \texttt{plot\_id}, \texttt{doc\_type\_id}, \texttt{issuing\_authority\_id}, \texttt{doc\_number}, \texttt{issue\_date}, \texttt{share\_type}, and \texttt{share\_ratio}.
    
    \item \textbf{Model:} Represents a machine learning model. Attributes include \texttt{id}, \texttt{name}, \texttt{version}, \texttt{description}, \texttt{created\_by}, \texttt{created\_at}, and \texttt{is\_active}.
    
    \item \textbf{Valuation:} Represents the output of a model prediction for a plot in a project. Attributes include \texttt{id}, \texttt{project\_id}, \texttt{plot\_id}, \texttt{model\_id}, \texttt{predicted\_price}, \texttt{created\_at}, and \texttt{created\_by}.
    
    \item \textbf{Project Plot:} Represents the association between a project and its plots. Attributes include \texttt{id}, \texttt{project\_id}, \texttt{plot\_id}, \texttt{valuation\_id}, and \texttt{note}.
    
    \item \textbf{Plot Feedback:} Stores user feedback on a plot's estimated price. Attributes include \texttt{id}, \texttt{plot\_id}, \texttt{user\_id}, \texttt{is\_price\_accepted}, \texttt{suggested\_price}, and \texttt{created\_at}.
    
    \item \textbf{Governorate, Town, Neighborhood:} Represent geographical hierarchy. Attributes include \texttt{id}, \texttt{code}, \texttt{name\_ar}, and relevant foreign keys.
    
    \item \textbf{Supporting Lookup Tables:} Includes \texttt{Admin Zoning}, \texttt{Ownership Document Type}, \texttt{Issuing Authority}, \texttt{Soil Type}, \texttt{Rock Type}, \texttt{Crop Type}, \texttt{Nuisance Type}, and \texttt{Restriction Type} with their respective codes and labels.
    
    \item \textbf{Plot Crops, Plot Nuisances, Plot Restrictions:} Represent details of crops, nuisances, and restrictions on a plot. Attributes include plot foreign key, type foreign key, and specific measurements such as \texttt{coverage\_pct}, \texttt{tree\_count}, or \texttt{severity}.
\end{enumerate}

\clearpage
\section{ER Diagram}
\begin{figure}[H]
  \captionsetup{position=top,labelformat=empty,labelsep=none}
  \centering
  \caption{Figure 3.4.1 ER Diagram}
  \vspace{6pt}
  \includegraphics[width=0.7\textheight]{images/ER.png}
  \label{fig:erd}
\end{figure}
\clearpage

\section{Database Description}

\subsection{Users}
\begin{itemize}[leftmargin=*,itemsep=2pt]
  \item \texttt{id}: integer; PK; auto-increment.
  \item \texttt{full\_name}: string; not-null.
  \item \texttt{email}: string; unique; not-null.
  \item \texttt{password}: string; not-null; $\geq$ 8 chars.
  \item \texttt{role}: enum \{ADMIN, ASSESSOR, DATA\_SCIENTIST\}; not-null.
  \item \texttt{created\_at}: datetime; not-null.
\end{itemize}

\subsection{Projects}
\begin{itemize}[leftmargin=*,itemsep=2pt]
  \item \texttt{id}: UUID; PK.
  \item \texttt{name}: string; not-null.
  \item \texttt{description}: string; not-null.
  \item \texttt{created\_by}: integer; FK $\rightarrow$ \texttt{users.id}; not-null.
  \item \texttt{created\_at}: datetime; not-null.
  \item \texttt{status}: enum \{ACTIVE, ARCHIVED\}; not-null.
\end{itemize}

\subsection{Plots (Land Parcels)}
\begin{itemize}[leftmargin=*,itemsep=2pt]
  \item \texttt{id}: UUID; PK.
  \item \texttt{plot\_code}: string; unique; not-null.
  \item \texttt{governorate\_id}: integer; FK $\rightarrow$ \texttt{governorates.id}; not-null.
  \item \texttt{town\_id}: integer; FK $\rightarrow$ \texttt{towns.id}; not-null.
  \item \texttt{neighborhood\_id}: integer; FK $\rightarrow$ \texttt{neighborhoods.id}; not-null.
  \item \texttt{area\_m2}: \texttt{decimal(12,2)}; $\geq 0$; not-null.
  \item \texttt{slope}: enum \{FLAT, SLIGHT, MODERATE, STEEP\}; not-null.
  \item \texttt{soil\_type\_id}: integer; FK $\rightarrow$ \texttt{soil\_type.id}; nullable.
  \item \texttt{rock\_type\_id}: integer; FK $\rightarrow$ \texttt{rock\_type.id}; nullable.
  \item \texttt{current\_land\_use}: enum \{RES, COM, AGR, IND, MIX, VACANT\}; not-null.
  \item \texttt{planned\_land\_use}: enum \{RES, COM, AGR, IND, MIX, VACANT\}; nullable.
  \item \texttt{far}: \texttt{decimal(6,3)}; $\geq 0$; nullable.
  \item \texttt{coverage\_ratio}: \texttt{decimal(4,2)}; 0–1; nullable.
  \item \texttt{is\_current}: boolean; not-null.
  \item \texttt{version\_no}: integer; not-null.
  \item \texttt{created\_by}: integer; FK $\rightarrow$ \texttt{users.id}; not-null.
  \item \texttt{created\_at}: datetime; not-null.
\end{itemize}

\subsection{Plot\_Documents}
\begin{itemize}[leftmargin=*,itemsep=2pt]
  \item \texttt{id}: integer; PK; auto-increment.
  \item \texttt{plot\_id}: UUID; FK $\rightarrow$ \texttt{plots.id}; not-null.
  \item \texttt{doc\_type\_id}: integer; FK $\rightarrow$ \texttt{ownership\_document\_type.id}; not-null.
  \item \texttt{issuing\_authority\_id}: integer; FK $\rightarrow$ \texttt{issuing\_authority.id}; not-null.
  \item \texttt{doc\_number}: string; not-null.
  \item \texttt{issue\_date}: date; not-null.
  \item \texttt{share\_type}: enum \{INDIVIDUAL, MUSHA\}; not-null.
  \item \texttt{share\_ratio}: string (e.g., “3/8”); nullable if \texttt{INDIVIDUAL}.
\end{itemize}

\subsection{Models}
\begin{itemize}[leftmargin=*,itemsep=2pt]
  \item \texttt{id}: UUID; PK.
  \item \texttt{name}: string; not-null.
  \item \texttt{version}: string; not-null.
  \item \texttt{description}: string; not-null.
  \item \texttt{created\_by}: integer; FK $\rightarrow$ \texttt{users.id}; not-null.
  \item \texttt{created\_at}: datetime; not-null.
  \item \texttt{is\_active}: boolean; not-null.
\end{itemize}

\subsection{Valuations}
\begin{itemize}[leftmargin=*,itemsep=2pt]
  \item \texttt{id}: UUID; PK.
  \item \texttt{project\_id}: UUID; FK $\rightarrow$ \texttt{projects.id}; not-null.
  \item \texttt{plot\_id}: UUID; FK $\rightarrow$ \texttt{plots.id}; not-null.
  \item \texttt{model\_id}: UUID; FK $\rightarrow$ \texttt{models.id}; not-null.
  \item \texttt{predicted\_price}: \texttt{decimal(12,2)}; not-null.
  \item \texttt{created\_at}: datetime; not-null.
  \item \texttt{created\_by}: integer; FK $\rightarrow$ \texttt{users.id}; not-null.
\end{itemize}

\subsection{Project\_Plots}
\begin{itemize}[leftmargin=*,itemsep=2pt]
  \item \texttt{id}: integer; PK; auto-increment.
  \item \texttt{project\_id}: UUID; FK $\rightarrow$ \texttt{projects.id}; not-null.
  \item \texttt{plot\_id}: UUID; FK $\rightarrow$ \texttt{plots.id}; not-null.
  \item \texttt{valuation\_id}: UUID; FK $\rightarrow$ \texttt{valuations.id}; nullable.
  \item \texttt{note}: string; optional short label.
\end{itemize}

\subsection{Plot\_Feedback}
\begin{itemize}[leftmargin=*,itemsep=2pt]
  \item \texttt{id}: integer; PK; auto-increment.
  \item \texttt{plot\_id}: UUID; FK $\rightarrow$ \texttt{plots.id}; not-null.
  \item \texttt{user\_id}: integer; FK $\rightarrow$ \texttt{users.id}; not-null.
  \item \texttt{is\_price\_accepted}: boolean; not-null.
  \item \texttt{suggested\_price}: \texttt{decimal(12,2)}; required if not accepted.
  \item \texttt{created\_at}: datetime; not-null.
\end{itemize}

\subsection{Governorates}
\begin{itemize}[leftmargin=*,itemsep=2pt]
  \item \texttt{id}: integer; PK; auto-increment.
  \item \texttt{code}: string; unique; not-null.
  \item \texttt{name\_ar}: string; not-null.
\end{itemize}

\subsection{Towns}
\begin{itemize}[leftmargin=*,itemsep=2pt]
  \item \texttt{id}: integer; PK; auto-increment.
  \item \texttt{governorate\_id}: integer; FK $\rightarrow$ \texttt{governorates.id}; not-null.
  \item \texttt{code}: string; unique; not-null.
  \item \texttt{name\_ar}: string; not-null.
\end{itemize}

\subsection{Neighborhoods}
\begin{itemize}[leftmargin=*,itemsep=2pt]
  \item \texttt{id}: integer; PK; auto-increment.
  \item \texttt{town\_id}: integer; FK $\rightarrow$ \texttt{towns.id}; not-null.
  \item \texttt{code}: string; unique; not-null.
  \item \texttt{name\_ar}: string; not-null.
\end{itemize}

\subsection{Admin\_Zoning}
\begin{itemize}[leftmargin=*,itemsep=2pt]
  \item \texttt{id}: integer; PK; auto-increment.
  \item \texttt{code}: string; unique; not-null.
  \item \texttt{label\_ar}: string; not-null.
\end{itemize}

\subsection{Ownership\_Document\_Type}
\begin{itemize}[leftmargin=*,itemsep=2pt]
  \item \texttt{id}: integer; PK; auto-increment.
  \item \texttt{code}: string; unique; not-null.
  \item \texttt{label\_ar}: string; not-null.
\end{itemize}

\subsection{Issuing\_Authority}
\begin{itemize}[leftmargin=*,itemsep=2pt]
  \item \texttt{id}: integer; PK; auto-increment.
  \item \texttt{code}: string; unique; not-null.
  \item \texttt{label\_ar}: string; not-null.
\end{itemize}

\subsection{Soil\_Type}
\begin{itemize}[leftmargin=*,itemsep=2pt]
  \item \texttt{id}: integer; PK; auto-increment.
  \item \texttt{code}: string; unique; not-null.
  \item \texttt{label\_ar}: string; not-null.
\end{itemize}

\subsection{Rock\_Type}
\begin{itemize}[leftmargin=*,itemsep=2pt]
  \item \texttt{id}: integer; PK; auto-increment.
  \item \texttt{code}: string; unique; not-null.
  \item \texttt{label\_ar}: string; not-null.
\end{itemize}

\subsection{Crop\_Type}
\begin{itemize}[leftmargin=*,itemsep=2pt]
  \item \texttt{id}: integer; PK; auto-increment.
  \item \texttt{code}: string; unique; not-null.
  \item \texttt{label\_ar}: string; not-null.
\end{itemize}

\subsection{Nuisance\_Type}
\begin{itemize}[leftmargin=*,itemsep=2pt]
  \item \texttt{id}: integer; PK; auto-increment.
  \item \texttt{code}: string; unique; not-null.
  \item \texttt{label\_ar}: string; not-null.
\end{itemize}

\subsection{Restriction\_Type}
\begin{itemize}[leftmargin=*,itemsep=2pt]
  \item \texttt{id}: integer; PK; auto-increment.
  \item \texttt{code}: string; unique; not-null.
  \item \texttt{label\_ar}: string; not-null.
\end{itemize}

\subsection{Plot\_Crops}
\begin{itemize}[leftmargin=*,itemsep=2pt]
  \item \texttt{id}: integer; PK; auto-increment.
  \item \texttt{plot\_id}: UUID; FK $\rightarrow$ \texttt{plots.id}; not-null.
  \item \texttt{crop\_type\_id}: integer; FK $\rightarrow$ \texttt{crop\_type.id}; not-null.
  \item \texttt{coverage\_pct}: \texttt{decimal(5,2)}; 0–100; nullable.
  \item \texttt{tree\_count}: integer; $\geq 0$; nullable.
\end{itemize}

\subsection{Plot\_Nuisances}
\begin{itemize}[leftmargin=*,itemsep=2pt]
  \item \texttt{id}: integer; PK; auto-increment.
  \item \texttt{plot\_id}: UUID; FK $\rightarrow$ \texttt{plots.id}; not-null.
  \item \texttt{nuisance\_type\_id}: integer; FK $\rightarrow$ \texttt{nuisance\_type.id}; not-null.
  \item \texttt{severity}: enum \{LOW, MEDIUM, HIGH\}; not-null.
\end{itemize}

\subsection{Plot\_Restrictions}
\begin{itemize}[leftmargin=*,itemsep=2pt]
  \item \texttt{id}: integer; PK; auto-increment.
  \item \texttt{plot\_id}: UUID; FK $\rightarrow$ \texttt{plots.id}; not-null.
  \item \texttt{restriction\_type\_id}: integer; FK $\rightarrow$ \texttt{restriction\_type.id}; not-null.
\end{itemize}


\section{Interfaces}
% --- UI Snapshots ---

\begin{figure}[H]
\centering
\captionsetup{position=top}
\caption{Figure 3.6.1 Account Registration}
\includegraphics[width=1\textwidth]{images/Register.png} 
\label{fig:account-registration}
\end{figure}

\begin{figure}[H]
\centering
\captionsetup{position=top}
\caption{Figure 3.6.2 Login Page}
\includegraphics[width=1\textwidth]{images/Login_1.png} 
\label{fig:login_page}
\end{figure}

\begin{figure}[H]
\centering
\captionsetup{position=top}
\caption{Figure 3.6.3 Forgot Password}
\includegraphics[width=1\textwidth]{images/Reset_password.png} 
\label{fig:forgot_password}
\end{figure}

\begin{figure}[H]
\centering
\captionsetup{position=top}
\caption{Figure 3.6.4 Home Page}
\includegraphics[width=1\textwidth]{images/Home_page.png} 
\label{fig:home_page}
\end{figure}

\begin{figure}[H]
\centering
\captionsetup{position=top}
\caption{Figure 3.6.5 Create New Project}
\includegraphics[width=1\textwidth]{images/Create_new_project.png} 
\label{fig:Create_New_Project}
\end{figure}

\begin{figure}[H]
\centering
\captionsetup{position=top}
\caption{Figure 3.6.6 View Projects}
\includegraphics[width=1\textwidth]{images/View_projects_Appraiser.png} 
\label{fig:View_Projects}
\end{figure}

\begin{figure}[H]
\centering
\captionsetup{position=top}
\caption{Figure 3.6.7 View Profile}
\includegraphics[width=1\textwidth]{images/Profile.png} 
\label{fig:View_Profile}
\end{figure}

\begin{figure}[H]
\centering
\captionsetup{position=top}
\caption{Figure 3.6.8 Edit Profile}
\includegraphics[width=1\textwidth]{images/Edit_profile.png} 
\label{fig:Edit_Profile}
\end{figure}

% --- Placeholders ---

\begin{figure}[H]
\centering
\captionsetup{position=top}
\caption{Figure 3.6.9 Select Project To Analyze}
\includegraphics[width=1\textwidth]{images/View_projects_Data_scientist.png}
\end{figure}

\begin{figure}[H]
\centering
\captionsetup{position=top}
\caption{Figure 3.6.10 Test Model Accuracy}
\includegraphics[width=1\textwidth]{images/Model_testing.png}
\end{figure}

\begin{figure}[H]
\centering
\captionsetup{position=top}
\caption{Figure 3.6.11 Review Feature Impact}
\includegraphics[width=1\textwidth]{images/Feature_impact_2.png}
\end{figure}

\begin{figure}[H]
\centering
\captionsetup{position=top}
\caption{Figure 3.6.12 Monitor Model Performance Over Time}
\includegraphics[width=1\textwidth]{images/Model_history.png}
\end{figure}

\begin{figure}[H]
\centering
\captionsetup{position=top}
\caption{Figure 3.6.13 Admin Dashboard}
\includegraphics[width=1\textwidth]{images/Admin_dashboard.png}
\end{figure}

\begin{figure}[H]
\centering
\captionsetup{position=top}
\caption{Figure 3.6.14 View And Manage Users}
\includegraphics[width=1\textwidth]{images/User_management.png}
\end{figure}

\begin{figure}[H]
\centering
\captionsetup{position=top}
\caption{Figure 3.6.15 View Admin Accounts}
\includegraphics[width=1\textwidth]{images/Admin_management_2.png}
\end{figure}

\begin{figure}[H]
\centering
\captionsetup{position=top}
\caption{Figure 3.6.16 Create Admin Account}
\includegraphics[width=1\textwidth]{images/Admin_management_2.png}
\end{figure}

\begin{figure}[H]
\centering
\captionsetup{position=top}
\caption{Figure 3.6.17 Manage Form Data}
\includegraphics[width=1\textwidth]{images/Manage_Data_1.png}
\end{figure}

\begin{figure}[H]
\centering
\captionsetup{position=top}
\caption{Figure 3.6.18 View System Logs}
\includegraphics[width=1\textwidth]{images/System_logs.png}
\end{figure}

\begin{figure}[H]
\centering
\captionsetup{position=top}
\caption{Figure 3.6.19 Manage Backups}
\includegraphics[width=1\textwidth]{images/Backup_management.png}
\end{figure}

\begin{figure}[H]
\centering
\captionsetup{position=top}
\caption{Figure 3.6.20 Create Activation Key}
\includegraphics[width=1\textwidth]{images/Activation_keys.png}
\end{figure}
% -------------------- Chapter 4 --------------------
\chaptertitlepage{System Implementation}

\section{Overview}
This chapter presents the implementation details of the Land Price Estimator system.
It describes the software environment, frontend and backend development, database technology,
system integration, and security mechanisms.
Each section focuses on how the theoretical design presented in previous chapters
was translated into a working and functional system.

\section{Software Environment}

\subsection{Programming Languages}

The Land Price Estimator system is built using a combination of programming and markup languages to efficiently handle both backend and frontend functionalities. The selected languages and their roles are described below:

\begin{itemize}[leftmargin=*]
  \item \textbf{Python:}\\
  Python is the main programming language used for the backend of the system. It is primarily utilized through the Django framework to implement the Model--View--Template (MVT) architecture, handle business logic, manage the database using the Object-Relational Mapper (ORM), and process user requests.

  \item \textbf{HTML (HyperText Markup Language):}\\
  HTML is used to structure the content of the web pages. It defines elements such as forms, links, buttons, and headings, providing the foundation for all frontend interfaces presented to users.

  \item \textbf{CSS (Cascading Style Sheets):}\\
  CSS is responsible for styling HTML elements, including layout, colors, fonts, spacing, and responsiveness. It ensures a consistent and visually appealing interface across different devices and browsers.

  \item \textbf{JavaScript:}\\
  JavaScript is used to implement interactive features on the frontend, such as dynamic form validation, responsive components, and user interface enhancements that improve overall usability.

  \item \textbf{Tailwind CSS:}\\
  Tailwind CSS is used alongside standard CSS to streamline styling and maintain consistency. It enables the use of predefined utility classes directly within HTML templates, speeding up development and reducing the need for extensive custom CSS.
\end{itemize}

\subsection{Web Framework}

The Land Price Estimator system is developed using the \textbf{Django} web framework. Django is a high-level Python framework that follows the \textbf{Model--View--Template (MVT)} architectural pattern and is designed to support rapid development, scalability, and secure web applications.

Django serves as the core backend framework of the system. It handles HTTP requests, implements business logic, manages database interactions, and renders dynamic web pages. The MVT architecture is clearly reflected in the project structure, where models define the data schema, views contain application logic, and templates generate the user interface.

The system adopts a \textbf{server-side rendering} approach using Django’s template engine. Dynamic content is generated on the server and delivered to users as fully rendered HTML pages, ensuring better security, simpler logic flow, and broad browser compatibility.

Django’s built-in \textbf{Object-Relational Mapping (ORM)} is used to interact with the database using Python objects instead of raw SQL queries. This abstraction improves code readability and maintainability while reducing the risk of SQL injection attacks.

User authentication and authorization are implemented using Django’s authentication framework with a \textbf{custom user model}. This allows the system to support multiple user roles, including normal users, data scientists, and administrators. Role-based access control is enforced using decorators such as \texttt{@login\_required}, and users are redirected to the appropriate system section after successful login based on their role.

Django Forms and ModelForms are used to handle user input, perform server-side validation, manage file uploads, and securely process sensitive data such as passwords. Password handling relies on Django’s built-in hashing mechanisms to ensure secure credential storage.

The project follows a \textbf{modular multi-application structure}, where each user role is implemented as a separate Django application. This design improves scalability, code organization, and long-term maintainability.

Overall, Django was selected due to its strong security features, clear architectural structure, built-in authentication system, and excellent support for data-driven web applications.

\subsection{Development Tools}

\begin{itemize}[leftmargin=*]
  \item \textbf{Integrated Development Environment (IDE):}\\
  Visual Studio Code (VS Code) was used as the primary IDE. It provides robust support for Python and Django development, including syntax highlighting, debugging tools, code formatting, and extensibility through plugins. Its lightweight and flexible design made it suitable for managing both frontend and backend code.

  \item \textbf{Version Control System:}\\
  Git was used for version control to track code changes and manage development history. The project repository was hosted on GitHub, enabling centralized code management, backups, and collaboration.

  \item \textbf{Database Management:}\\
  SQLite was used during development as the default database system provided by Django. It is lightweight and easy to configure. Database interactions were handled using Django’s ORM without writing raw SQL queries.

  \item \textbf{Web Browser and Testing Tools:}\\
  Modern web browsers such as Brave and Comet were used for frontend testing and debugging. Built-in developer tools were used to inspect HTML elements, debug JavaScript, and test responsiveness across different screen sizes.

  \item \textbf{Machine Learning Development Environment:}\\
  Google Colab was used to develop, train, and validate the machine learning model for land price prediction. Using Python libraries such as scikit-learn, a Decision Tree Regression model was trained on historical land data. The trained model was then exported and integrated into the Django backend to provide real-time price predictions based on user input.
\end{itemize}

\subsection{Supporting Libraries}

In addition to the main frameworks and tools, several supporting libraries were used to facilitate backend development, form handling, authentication, and machine learning:

\begin{itemize}[leftmargin=*]
  \item \textbf{Django Built-in Libraries:}\\
  Django’s built-in libraries were used for authentication, authorization, request handling, and form processing. These utilities ensure that only authorized users can access restricted system features.

  \item \textbf{Django Forms:}\\
  Django’s ModelForm system was used to automatically generate forms based on database models. This simplifies form creation, enforces server-side validation, and ensures consistency between user input and stored data.

  \item \textbf{Machine Learning Libraries:}\\
  Scikit-learn was used to implement the Decision Tree Regression algorithm. The \texttt{train\_test\_split} function was used to divide the dataset into training and testing sets to ensure proper model evaluation.

  \item \textbf{Data Processing Library:}\\
  Pandas was used for data manipulation and preprocessing during model development. It was utilized to load datasets, handle missing values, and prepare structured data suitable for training the regression model.
\end{itemize}


\section{Frontend Implementation}

\subsection{General Frontend Design}
The frontend of the Land Price Estimator system is a web-based interface designed to be simple, clear, and easy to use. The system is accessed through a web browser and does not require any additional software, allowing users to work on different devices.

The frontend uses server-side rendering with Django templates, where pages are generated on the server and sent to the user as complete HTML pages. This approach ensures consistent display, good performance, and smooth integration with backend features such as authentication and form handling.

A base template is used to maintain a consistent layout across all pages, including shared components like navigation bars and common styles. Individual pages extend this base template to display their specific content.

The user interface is styled using CSS and Tailwind CSS utility classes, which help create a responsive and consistent design with minimal custom styling. JavaScript is used only to support basic interactive features and improve user experience when needed.

Overall, the frontend design focuses on usability and clarity, enabling users to navigate the system easily and perform land price estimation tasks efficiently.

\subsection{Authentication Interfaces}
The system provides authentication interfaces that control access to the platform and ensure that only authorized users can register and log in. These interfaces include the registration page and the login page.

During registration, users are required to enter an activation code provided by the system administrator. This activation code determines the user role (such as appraiser, administrator, or data scientist) and controls which part of the system the user can access after registration. This approach ensures controlled user onboarding and role-based access from the moment the account is created.

\begin{figure}[H]
\centering
\captionsetup{position=top}
\caption{Figure 4.3.1 Registration Page}
\includegraphics[width=\textwidth]{images/Figure 4.3.1 Registration page.png}
\label{fig:frontend-registration}
\end{figure}

The login interface allows registered users to authenticate using their email and password. After successful login, users are automatically redirected to their respective dashboard based on their assigned role. If authentication fails, the system displays clear error messages to guide the user.

\begin{figure}[H]
\centering
\captionsetup{position=top}
\caption{Figure 4.3.2 Login Page}
\includegraphics[width=\textwidth]{images/Figure 4.3.2 Login page.png}
\label{fig:frontend-login}
\end{figure}

Both authentication interfaces are designed to be simple and user-friendly, with clear input fields, validation feedback, and consistent styling. This design helps users complete authentication tasks easily while maintaining security and proper access control.

\subsection{User Dashboard (Home Page)}
After successful authentication, users are redirected to their dashboard, which serves as the main entry point to the system. The dashboard provides an overview of the user’s activity and quick access to the system’s core features.

The interface includes clear navigation options that allow the user to:
\begin{itemize}[leftmargin=*]
  \item View existing land price estimation projects.
  \item Create a new project (land price estimation).
  \item Access their profile and account settings.
\end{itemize}

The dashboard layout is designed to be simple and role-aware, meaning each user sees options relevant to their assigned role (e.g., appraiser or data scientist). This improves usability and prevents access to unauthorized features.

\begin{figure}[H]
\centering
\captionsetup{position=top}
\caption{Figure 4.3.3 Home Page}
\includegraphics[width=\textwidth]{images/Figure 4.3.3 Homepage.png}
\label{fig:frontend-homepage}
\end{figure}

Overall, the dashboard acts as a centralized control panel, enabling users to efficiently navigate the system and manage their land price estimation projects.

\subsection{Project Management Interface (View Projects)}
The Project Management Interface allows users to view and manage all previously created land price estimation projects in an organized manner. This page displays a list of the user’s projects, including key information such as project name, creation date, and current status.

To improve usability and efficiency, the interface provides several tools that help users quickly locate specific projects:
\begin{itemize}[leftmargin=*]
  \item Search functionality to find projects by name or keyword.
  \item Filtering options to narrow results based on project status or other attributes.
  \item Sorting options to order projects by date.
\end{itemize}

Each project entry includes actions that allow the user to open the project, review its details, or continue working on it. This interface helps users manage multiple estimations efficiently without confusion.

\begin{figure}[H]
\centering
\captionsetup{position=top}
\caption{Figure 4.3.4 Project Management Interface}
\includegraphics[width=\textwidth]{images/Figure 4.3.4 Project Management Interface.png}
\label{fig:frontend-project-management}
\end{figure}

The project management page plays a crucial role in organizing user data and ensuring easy access to land price estimation records.

\subsection{New Project / Land Price Estimation Interface}
The New Project / Land Price Estimation Interface allows users to create a new land evaluation project by entering detailed information about a land parcel. This page is centered around a structured form designed to collect all required attributes needed for price estimation.

The form includes fields related to land characteristics such as location, area, zoning type, and other relevant parameters. Input validation is applied to ensure data accuracy before submission.

Users are provided with two main actions:
\begin{itemize}[leftmargin=*]
  \item \textbf{Save Project:} Stores the entered data without performing price estimation, allowing users to complete or modify the project later.
  \item \textbf{Estimate Price:} Submits the land data to the system, triggers the price estimation process, and automatically saves the project along with the estimated land price.
\end{itemize}

The interface is designed to be simple, guiding users step-by-step through the data entry process while minimizing errors. Clear labels and structured layout help ensure a smooth user experience.

\begin{figure}[H]
\centering
\captionsetup{position=top}
\caption{Figure 4.3.5 New Project Form}
\includegraphics[width=\textwidth]{images/Figure 4.3.5 New project form.png}
\label{fig:frontend-new-project-form}
\end{figure}

This interface represents the core functionality of the system, connecting user input with the land price estimation process.



% =========================
% 4.4 Backend Implementation
% =========================

\section{Backend Implementation}

\subsection{Overview of Backend Architecture}

The backend of the Land Price Estimator system is implemented using the Django web framework and follows the Model–View–Template (MVT) architectural pattern. All core application logic, including authentication, project management, role-based access control, and machine learning inference, is handled on the server side.

The backend is responsible for:
\begin{itemize}
  \item Managing user authentication and authorization
  \item Handling project creation, storage, and retrieval
  \item Enforcing user ownership over projects
  \item Integrating a trained machine learning model for land price estimation
  \item Serving validated data to frontend templates
\end{itemize}

The system does not rely on external API servers; all logic is processed internally within Django.

\subsection{Application Structure and Separation of Concerns}

The backend is organized into multiple Django applications, each responsible for a specific domain:
\begin{itemize}
  \item \textbf{Users\_Handling\_App}\\
  Handles authentication, registration, activation codes, and user redirection based on roles.

  \item \textbf{normal\_user}\\
  Represents the main application used by land appraisers. It includes project creation, project listing, filtering, and machine learning–based price estimation.

  \item \textbf{data\_scientist \& admin apps}\\
  Planned for future work. Currently, administrative tasks are handled via Django’s built-in admin panel.
\end{itemize}

This modular structure improves maintainability, scalability, and clarity of responsibilities.

\begin{figure}[H]
  \centering
  \caption{Figure 4.4.1 Project structure}
  \includegraphics[width=0.9\linewidth]{images/Figure 4.4.1 Project structure.png}
\end{figure}

\begin{figure}[H]
  \centering
  \caption{Figure 4.4.2 Normal user side app structure}
  \includegraphics[width=0.9\linewidth]{images/Figure 4.4.2 Normal user side app structure.png}
\end{figure}

\subsection{Data Models and Database Design}

\textbf{Custom User Model}\\
A custom user model is implemented by extending Django’s \texttt{AbstractUser}, replacing the username with email-based authentication. Each user is assigned a role that determines system access.

Key features:
\begin{itemize}
  \item Email-based login
  \item Role-based user types (Land Appraiser, Data Scientist, Admin)
  \item Custom user manager for controlled user creation
\end{itemize}

\begin{figure}[H]
  \centering
  \caption{Figure 4.4.3 Custom user model}
  \includegraphics[width=0.9\linewidth]{images/Figure 4.4.3 Custom user model.png}
\end{figure}

\textbf{Project Model}\\
The Project model represents a land valuation request created by a land appraiser. Each project is associated with exactly one user, while each user may have multiple projects.

Key attributes include:
\begin{itemize}
  \item Governorate
  \item Land size
  \item Land type
  \item Political classification
  \item Project status (draft or completed)
  \item Creation date
\end{itemize}

This design enforces ownership and allows secure filtering of projects per user.

\begin{figure}[H]
  \centering
  \caption{Figure 4.4.4 Project model}
  \includegraphics[width=0.9\linewidth]{images/Figure 4.4.4 Project model.png}
\end{figure}

\subsection{Forms and Server-Side Validation}

Django ModelForms are used to handle structured user input and enforce validation rules.

\begin{itemize}
  \item \textbf{UserForm}\\
  Used for profile updates, including secure password change with validation.

  \item \textbf{ProjectForm}\\
  Used to create new land valuation projects. The form maps directly to the Project model and ensures data consistency.
\end{itemize}

Server-side validation ensures:
\begin{itemize}
  \item Required fields are enforced
  \item Invalid data is rejected
  \item Security is maintained regardless of frontend behavior
\end{itemize}

\begin{figure}[H]
  \centering
  \caption{Figure 4.4.5 Project form}
  \includegraphics[width=0.9\linewidth]{images/Figure 4.4.5 Project form.png}
\end{figure}

\begin{figure}[H]
  \centering
  \caption{Figure 4.4.6 User form}
  \includegraphics[width=0.9\linewidth]{images/Figure 4.4.6 User form.png}
\end{figure}

\subsection{Project Creation and State Management}

When a user submits the project creation form, the backend distinguishes between two actions:
\begin{itemize}
  \item \textbf{Save as Draft}\\
  Stores the project without invoking the machine learning model.

  \item \textbf{Estimate Price}\\
  Triggers the ML prediction process and marks the project as completed.
\end{itemize}

This logic is handled within the Django view by detecting the submitted button name.

\begin{figure}[H]
  \centering
  \caption{Figure 4.4.7 newProject view}
  \includegraphics[width=0.9\linewidth]{images/Figure 4.4.7 newProject view.png}
\end{figure}

\subsection{Machine Learning Model Integration}

A Decision Tree Regression model was trained externally using Google Colab and exported using \texttt{joblib}. The trained model is loaded into Django and used for real-time inference.

ML Integration Workflow:
\begin{enumerate}
  \item Project data is validated and saved
  \item Relevant features are extracted
  \item Data is preprocessed to match training format
  \item The trained model predicts land price
  \item Prediction is stored in the database
  \item Project status is updated to completed
\end{enumerate}

The ML logic is isolated in a dedicated module to maintain separation between business logic and machine learning inference.

\begin{figure}[H]
  \centering
  \caption{Figure 4.4.8 Import the ML model}
  \includegraphics[width=0.9\linewidth]{images/Figure 4.4.8 Import the ML model.png}
\end{figure}

\begin{figure}[H]
  \centering
  \caption{Figure 4.4.9 Prediction function}
  \includegraphics[width=0.9\linewidth]{images/Figure 4.4.9 Prediction function.png}
\end{figure}

\subsection{Access Control and Data Security}

The backend enforces strict access control rules:
\begin{itemize}
  \item Users can only view and manage their own projects
  \item Authentication is required for all protected views
  \item Role-based redirection ensures users are sent to the correct interface
  \item Django’s built-in authentication and session management mechanisms are used
\end{itemize}

Filtering and sorting operations are processed on the server side to prevent unauthorized data exposure.

\begin{figure}[H]
  \centering
  \caption{Figure 4.4.10 Filtering logic in the view}
  \includegraphics[width=0.9\linewidth]{images/Figure 4.4.10 Filtering logic in the view.png}
\end{figure}

% ==================== 4.5 ====================
\section{Database Technology Implementation}

This section explains the database technology used in the Land Price Estimator system.
It describes the selected database engine, the data access methodology, core entities,
relationships, validation mechanisms, and schema management techniques.
The database layer plays a critical role in maintaining data consistency,
integrity, and reliability throughout the system.

\subsection{Database Engine Selection}
The system uses \textbf{SQLite} as its database management system.
SQLite was selected due to its lightweight architecture, ease of deployment,
and seamless integration with the Django framework.
As a file-based database engine, SQLite does not require a separate database server,
which makes it suitable for academic projects and prototype systems.

The use of SQLite allows rapid development, straightforward testing,
and efficient storage of structured relational data during the project lifecycle.

\subsection{Database Access Using Django ORM}
All interactions with the database are handled through the \textbf{Django Object-Relational Mapping (ORM)} layer.
The ORM abstracts direct SQL queries and enables developers to manipulate database records
using Python objects and classes.

Each database table is represented as a Django model, where fields correspond to table columns,
and relationships are expressed using foreign key definitions.
This approach improves code readability, simplifies maintenance,
and reduces the risk of security vulnerabilities such as SQL injection.

\begin{figure}[H]
\centering
\captionsetup{position=top}
\caption{Figure 4.5.1 Django Models Definition}
\includegraphics[width=\textwidth]{images/Figure 4.5.1 Django Models Definition.png}
\end{figure}

\subsection{Core Database Entities}
The database schema is centered around two main entities:

\textbf{Users} store authentication and role-related information for system participants,
including administrators, land appraisers, and data scientists.

\textbf{Projects} represent land valuation tasks created by users.
Each project is associated with its creator, ensuring ownership tracking
and controlled access to project data.

This structure supports clear separation of responsibilities
and organized storage of estimation records.

\subsection{Primary Keys and Relationships}
All tables use \textbf{BigAutoField} as their primary key,
which provides an auto-incremented integer identifier for each record.
This ensures uniqueness and efficient indexing.

Foreign keys are used to establish relationships between tables,
such as linking projects to users.
These relationships enforce \textbf{referential integrity},
ensuring that dependent records always reference valid parent records.

UUID values are not used as primary keys.
Instead, UUIDs are generated only for activation codes,
providing secure and unpredictable identifiers for account activation purposes.

\begin{figure}[H]
\centering
\captionsetup{position=top}
\caption{Figure 4.5.2 Primary and Foreign Key Relationships}
\includegraphics[width=0.9\textwidth]{images/Figure 4.5.2 Primary and Foreign Key Relationships.png}
\end{figure}

\subsection{Data Validation and Constraints}
Data integrity is ensured through multiple validation mechanisms at the model level.
These include enforcing correct data types,
restricting null values for essential fields,
and applying unique constraints on sensitive attributes such as email addresses.

Primary and foreign key constraints further guarantee the consistency of relational data.
This validation strategy minimizes invalid entries
and preserves database reliability throughout system operation.

\subsection{Database Migrations}
The database schema is managed using Django’s built-in migration framework.
Migrations allow controlled and incremental updates to the database structure
while preserving existing data.

Commands such as \texttt{makemigrations} and \texttt{migrate}
are used to synchronize model definitions with the SQLite database schema.

\begin{figure}[H]
\centering
\captionsetup{position=top}
\caption{Figure 4.5.3 Database Migration Process}
\includegraphics[width=0.85\textwidth]{images/Figure 4.5.3 Database Migration Process.png}
\end{figure}

\subsection{Dataset Usage}
The current implementation relies on \textbf{synthetic data}
generated for development, testing, and model evaluation.
This approach enables controlled experimentation
without dependency on sensitive or unavailable real-world datasets.

Future work includes integrating real land data
through an official land registry API,
which would allow continuous updates
and enhance prediction accuracy and system realism.

\subsection{Design Considerations}
The combination of SQLite and Django ORM provides a balanced solution
between simplicity, maintainability, and performance.
The database design allows smooth migration
to more advanced database engines in future deployments
without major architectural changes.

% ==================== 4.6 ====================
\section{System Integration}

System integration ensures that all components of the Land Price Estimator system work together seamlessly. This includes the interaction between the frontend, backend, database, machine learning model, and email service. Proper integration allows users to submit data, receive estimations, and manage projects efficiently while maintaining security and role-based access.

\subsection{Integration of Frontend and Backend}

The frontend interface communicates with the backend through Django views using server-side rendering. When a user interacts with the interface, such as submitting a new project form, the following occurs:

\begin{enumerate}[leftmargin=*]
    \item The form data is sent via a POST request to the corresponding Django view (newProject).
    \item The view validates the data using Django forms (ProjectForm) and performs the requested action, either saving as a draft or estimating the land price.
    \item After processing, the user is redirected to the appropriate page (e.g., project list or dashboard).
\end{enumerate}

\subsection{Integration with the Machine Learning Model}

The system integrates the Decision Tree Regression model for estimating land prices:

\begin{enumerate}[leftmargin=*]
    \item Upon selecting the “Estimate Price” action, the backend view prepares the input data from the form.
    \item The input data is sent to the predict\_land\_price() function, which uses the model\_loader to load the exported ML model.
    \item The model predicts the price and the result is stored in the Project model in the database.
\end{enumerate}

Error handling is implemented to manage cases where the ML model is missing or input data is invalid, ensuring users receive a user-friendly message instead of a system error.

\subsection{Role-Based Access Control Integration}

User roles are integrated to ensure security and appropriate access:

\begin{itemize}[leftmargin=*]
    \item Normal Users (Appraisers) can create projects, view their projects, and estimate land prices.
    \item Data Scientists (planned for future work) will access projects for analysis and model improvement.
    \item Administrators manage users, activation codes, and project oversight via the Django admin panel.
\end{itemize}

The backend enforces these restrictions using Django authentication, the User model, and role checks in views.

\begin{figure}[H]
    \centering
    \caption{Figure 4.6.1 URL routing for different user roles}
    \vspace{6pt}
    \includegraphics[width=\textwidth]{images/Figure 4.6.1 URL routing for different user roles.png}
\end{figure}

\subsection{Integration of Email Service}

The email service is integrated for password resets:

\begin{enumerate}[leftmargin=*]
    \item Users can request a password reset through the frontend.
    \item Django’s built-in password reset views handle the request and send an email using the configured SMTP backend.
    \item Users follow the link in the email to reset their password securely.
\end{enumerate}

\begin{figure}[H]
    \centering
    \caption{Figure 4.6.2 URLs configuration for password reset}
    \vspace{6pt}
    \includegraphics[width=\textwidth]{images/Figure 4.6.2 URLs configuration for password reset.png}
\end{figure}

\begin{figure}[H]
    \centering
    \caption{Figure 4.6.3 Email service integration in settings.py}
    \vspace{6pt}
    \includegraphics[width=\textwidth]{images/Figure 4.6.3 Email service integration in settings.py.png}
\end{figure}

\subsection{Data Flow and Database Integration}

All user actions and generated data are stored in the Project model, while user credentials and roles are managed in the Users\_Handling\_App models. The integration ensures:

\begin{itemize}[leftmargin=*]
    \item Data from forms is validated before saving.
    \item ML predictions are stored alongside the project data.
    \item Only authorized users can view or modify their own projects.
\end{itemize}

Data flow:

\begin{center}
User input $\rightarrow$ View (processing \& validation) $\rightarrow$ Database $\rightarrow$ Dashboard / Project List update
\end{center}

% ==================== 4.7 ====================
% ==================== 4.7 ====================
\section{Security Implementation}
This section describes the security mechanisms applied in the system to protect user data, prevent unauthorized access, and ensure safe interaction between system components. The implementation relies on Django’s built-in security features combined with application-level controls.

\subsection{Authentication and Access Control}
User authentication is handled using Django’s built-in authentication system. Only authenticated users can access protected views such as project creation, project listing, and land price estimation.
Access to system functionality is restricted based on user roles (e.g., normal user, administrator, data scientist). Each user can only access views and data relevant to their role, preventing unauthorized actions or data exposure.

\begin{figure}[H]
\centering
\captionsetup{position=top}
\caption{Figure 4.7.1 Authentication form}
\includegraphics[width=0.9\textwidth]{images/Figure 4.7.1 Authentication form.png}
\label{fig:security-auth-form}
\end{figure}

\begin{figure}[H]
\centering
\captionsetup{position=top}
\caption{Figure 4.7.2 Authentication in loginPage view}
\includegraphics[width=0.9\textwidth]{images/Figure 4.7.2 Authentication in loginPage view.png}
\label{fig:security-auth-loginpage}
\end{figure}

\subsection{Authorization and Data Isolation}
To ensure data privacy, users are restricted to viewing and managing only their own projects. Database queries are filtered at the view level to return records associated exclusively with the currently authenticated user.
This approach prevents the user from attempting to access or manipulate another user’s data.

\begin{figure}[H]
\centering
\captionsetup{position=top}
\caption{Figure 4.7.3 User-based query filtering in viewProjects view}
\includegraphics[width=0.9\textwidth]{images/Figure 4.7.3 User-based query filtering in viewProjects view.png}
\label{fig:security-query-filtering}
\end{figure}

\subsection{Password Management and Account Recovery}
Passwords are securely stored using Django’s hashing framework, which applies industry-standard hashing algorithms. The system also supports password recovery through email-based password reset functionality, implemented using Django’s built-in password reset views.
This ensures that users can safely recover access to their accounts without exposing sensitive credentials.
In addition to password recovery via email, authenticated users are allowed to change their passwords from the profile editing interface. To enhance security, users are required to provide their current password before setting a new one. The system validates the old password server-side before applying any changes, preventing unauthorized password updates.

\begin{figure}[H]
\centering
\captionsetup{position=top}
\caption{Figure 4.7.4 Change Password section in edit\_profile page}
\includegraphics[width=0.9\textwidth]{images/Figure 4.7.4 Change Password section in edit_profile page.png}
\label{fig:security-change-password}
\end{figure}

\begin{figure}[H]
\centering
\captionsetup{position=top}
\caption{Figure 4.7.5 Reset password via email}
\includegraphics[width=0.9\textwidth]{images/Figure 4.7.5 Reset password via email.png}
\label{fig:security-reset-password-email}
\end{figure}

\subsection{Form Validation and Input Protection}
All user inputs are validated on the server side using Django forms and model validation. This includes checking required fields, validating data formats, and enforcing business rules before any data is stored in the database.
This reduces the risk of invalid data entry and helps protect against common attacks such as malformed input and unintended data manipulation.

\begin{figure}[H]
\centering
\captionsetup{position=top}
\caption{Figure 4.7.6 ModelForm validation}
\includegraphics[width=0.9\textwidth]{images/Figure 4.7.6 ModelForm validation.png}
\label{fig:security-modelform-validation}
\end{figure}

\subsection{Cross-Site Request Forgery (CSRF) Protection}
All POST requests in the system are protected using Django’s CSRF middleware. CSRF tokens are embedded in all forms and verified on submission, ensuring that requests originate from legitimate users and not malicious third-party sources.
This protection is enabled by default and requires no additional configuration beyond proper template usage.

\begin{figure}[H]
\centering
\captionsetup{position=top}
\caption{Figure 4.7.7 CSRF token usage example}
\includegraphics[width=0.9\textwidth]{images/Figure 4.7.7 CSRF token usage example.png}
\label{fig:security-csrf-token}
\end{figure}

\subsection{Error Handling and Safe Failure}
The system includes structured error handling to prevent application crashes and avoid exposing sensitive internal information to users. For example, failures in machine learning prediction or missing model files are caught and handled gracefully, displaying user-friendly messages instead of system errors.
This approach improves both system reliability and security by preventing information leakage.

\begin{figure}[H]
\centering
\captionsetup{position=top}
\caption{Figure 4.7.8 Exception handling during land price estimation}
\includegraphics[width=0.9\textwidth]{images/Figure 4.7.8 Exception handling during land price estimation.png}
\label{fig:security-exception-handling}
\end{figure}

\subsection{Deployment and Future Security Enhancements}
The current system is developed in a local environment; however, it is designed to be production-ready. Future deployment plans include enabling HTTPS, secure email credentials, environment-based configuration, and stricter logging and monitoring.
These measures will further strengthen system security when deployed in a production environment.

% -------------------- Chapter 5 --------------------
\chaptertitlepage{Testing}

\section{Unit Testing}

Unit testing was conducted to verify the correctness of individual system components in isolation, particularly the data model responsible for storing land project information. Django’s built-in testing framework was used to ensure reliability, repeatability, and isolation from production data.

\subsection{Project Model Unit Test}

A unit test was implemented to validate the correct creation of a Project instance. This test focuses on confirming that essential attributes—such as the associated user, land size, governorate, and project status—are correctly stored in the database.

Figure 5.1.1 shows the unit test implementation within the tests.py file of the Normal\_User\_Side application. The test uses Django’s TestCase class, which automatically provides a temporary test database and a clean environment for each test run.

\begin{figure}[H]
    \centering
    \caption{Figure 5.1.1 Unit test implementation for the Project model}
    \vspace{6pt}
    \includegraphics[width=\textwidth]{images/Figure 5.1.1 Unit test implementation for the Project model.png}
\end{figure}

The test setup creates a sample user and a project instance, then applies assertions to verify that:

\begin{itemize}[leftmargin=*]
    \item The project is correctly linked to the authenticated user
    \item The project fields contain the expected values
    \item The default project status is assigned correctly
\end{itemize}

This approach ensures that the core data model behaves as intended before being integrated with views, forms, or machine learning components.

\subsection{Test Execution and Results}

The unit test was executed using Django’s test runner. Figure 5.1.2 illustrates the successful execution output, where Django automatically creates a temporary test database, runs the test, and then destroys the database after completion.

\begin{figure}[H]
    \centering
    \caption{Figure 5.1.2 Unit test result}
    \vspace{6pt}
    \includegraphics[width=\textwidth]{images/Figure 5.1.2 Unit test result.png}
\end{figure}

The successful test result confirms that the Project model functions correctly in isolation and that the system’s foundational data layer is stable.



\section{Integration Testing}

Integration testing ensures that different parts of the application work together correctly. For this project, the main focus was on the interaction between the project creation view, the ML model, and the database.

We created a simplified integration test in Normal\_User\_Side/tests.py that covers the following scenarios:

\begin{enumerate}[leftmargin=*]
    \item Creating a project as a draft.
    \item Creating a project and performing ML price estimation.
    \item Handling ML prediction failures gracefully.
    \item Ensuring that users can only see their own projects in the project list.
\end{enumerate}

\subsection{Test Code}

\begin{figure}[H]
    \centering
    \caption{Figure 5.2.1 Project Integration Test}
    \vspace{6pt}
    \includegraphics[width=\textwidth]{images/Figure 5.2.1 Project Integration Test.png}
\end{figure}

\subsection{Explanation}

The test class ProjectIntegrationTest logs in a test user, submits project data to the new-project view, and checks the following:

\begin{itemize}[leftmargin=*]
    \item Whether the project is correctly saved as draft or completed.
    \item If the ML prediction is performed and the estimated price is saved.
    \item Whether errors in ML prediction are handled properly and do not break the flow.
    \item Whether the project list view only shows projects belonging to the logged-in user.
\end{itemize}

This ensures that the end-to-end workflow of project creation and listing works as intended.

\subsection{Test Results}

\begin{figure}[H]
    \centering
    \caption{Figure 5.2.2 Integration Test Result}
    \vspace{6pt}
    \includegraphics[width=\textwidth]{images/Figure 5.2.2 Integration Test Result.png}
\end{figure}

\subsection{Explanation}

The results confirm that all integration scenarios are working:

\begin{itemize}[leftmargin=*]
    \item Draft projects are saved correctly.
    \item ML estimation is integrated successfully.
    \item Error handling works as expected.
    \item User-based project filtering is functional.
\end{itemize}
% ==================== 5.3 ====================
\section{End-to-End (E2E) Testing}
End-to-end testing evaluates the system as a complete unit by simulating real user interactions from initial access to final output. This testing approach ensures that all system layers—user interface, backend logic, authentication, database operations, and machine learning integration—work together seamlessly.
In this project, end-to-end testing was conducted manually, reflecting realistic usage scenarios performed by a normal user (land appraiser).

\subsection{User Registration and Authentication}
The testing process begins with the registration of a new normal user through the system’s registration interface.

\begin{figure}[H]
\centering
\captionsetup{position=top}
\caption{Figure 5.3.1 Registration}
\includegraphics[width=0.9\textwidth]{images/Figure 5.3.1 Registration.png}
\label{fig:e2e-registration}
\end{figure}

Explanation:
The user creates an account by providing required information and upon successful registration, the user will be redirected to the login page with a success message and will be able to authenticate using the login interface.

\begin{figure}[H]
\centering
\captionsetup{position=top}
\caption{Figure 5.3.2 Registration successful}
\includegraphics[width=0.9\textwidth]{images/Figure 5.3.2 Registration successful.png}
\label{fig:e2e-registration-success}
\end{figure}

\subsection{Dashboard Access and Navigation}
Once authenticated, the user gains access to the main dashboard.

\begin{figure}[H]
\centering
\captionsetup{position=top}
\caption{Figure 5.3.3 Dashboard}
\includegraphics[width=0.9\textwidth]{images/Figure 5.3.3 Dashboard.png}
\label{fig:e2e-dashboard}
\end{figure}

Explanation:
The dashboard serves as the central interface for project management, allowing the user to create new projects or view previously created ones. This confirms correct session handling and access control.

\subsection{New Project Creation}
The user initiates the creation of a new land price estimation project by navigating to the “New Project” interface.

\begin{figure}[H]
\centering
\captionsetup{position=top}
\caption{Figure 5.3.4 New Project form filled with data}
\includegraphics[width=0.9\textwidth]{images/Figure 5.3.4 New Project form filled with data.png}
\label{fig:e2e-new-project-form}
\end{figure}

Explanation:
 The user fills in land-related attributes such as governorate, land size, land type, and political classification. These inputs represent the real-world data required for land price estimation.

\subsection{Project Submission (Draft vs. Estimation)}
At submission time, the user can choose between two actions:

\begin{itemize}[leftmargin=*]
    \item Save as Draft: Stores the project without running the machine learning model.
    \item Estimate Price: Triggers the machine learning prediction process. 
\end{itemize}
\begin{figure}[H]
\centering
\captionsetup{position=top}
\caption{Figure 5.3.5 Save and Estimate buttons}
\includegraphics[width=0.9\textwidth]{images/Figure 5.3.5 Save and Estimate buttons.png}
\label{fig:e2e-save-estimate-buttons}
\end{figure}

Explanation:
 This step validates conditional backend logic based on user action. Draft projects are stored with a “draft” status, while estimated projects are marked as “completed” after prediction.

\subsection{Project Viewing and Verification}
After submission, the user is redirected to the projects list page.

\begin{figure}[H]
\centering
\captionsetup{position=top}
\caption{Figure 5.3.6 View the created project in the list of projects}
\includegraphics[width=0.9\textwidth]{images/Figure 5.3.6 View the created project in the list of projects.png}
\label{fig:e2e-project-list}
\end{figure}

Explanation:
 The projects page displays all projects created by the authenticated user. This confirms correct database persistence, user-based data filtering, and real-time dashboard updates.

Summary:
The successful completion of these end-to-end scenarios confirms that the system satisfies its functional and user requirements. All critical user actions—from account creation and authentication to land price estimation and project review—behaved as expected, thereby validating the system from an acceptance testing perspective.

% -------------------- References --------------------
\chapter*{References}
\addcontentsline{toc}{chapter}{References}

\begin{enumerate}[label={[\arabic*]}]

\item J.~Starmer, ``Regression Trees, Clearly Explained!!!,'' \textit{StatQuest with Josh Starmer}, YouTube, 2020. [Online]. Available: \url{https://youtu.be/g9c66TUylZ4?si=0V35RgYavNRFB-yx}. Accessed: Jul.~3,~2025.

\item J.~Starmer, ``CatBoost Part 1: Ordered Target Encoding,'' \textit{StatQuest with Josh Starmer}, YouTube, 2023. [Online]. Available: \url{https://youtu.be/KXOTSkPL2X4?si=Iq890z0lxjhSlImH}. Accessed: Jul.~3,~2025.

\item J.~Starmer, ``CatBoost Part 2: Building and Using Trees,'' \textit{StatQuest with Josh Starmer}, YouTube, 2023. [Online]. Available: \url{https://youtu.be/3Bg2XRFOTzg?si=SUU2vVzNxbofFelA}. Accessed: Jul.~3,~2025.

\end{enumerate}

\end{document}
